\documentclass[12pt,a4paper]{journal}
\usepackage[margin=1in]{geometry}

\usepackage[utf8]{inputenc}
\usepackage[T1]{fontenc}
\usepackage{indentfirst}
\usepackage{amsmath}
\usepackage{amsfonts}
\usepackage{amssymb}
\usepackage{caption}
\usepackage{subcaption}
\usepackage{graphicx}
\usepackage{siunitx}
\sisetup{range-phrase=--, range-units=single}

\graphicspath{{img/}}
\usepackage[numbers, square]{natbib}


\begin{document}

\subsection{Physiology of the retina}

Images to include:
\begin{itemize}
\item diagram of the organization of the eye (overall structure)
\item diagram/picture of the organization of the retinal layers
\item images of healthy retinal scans (structural/angiogram)
\end{itemize}


\subsubsection{In health}

\begin{itemize}
\item organization of the eye: iris/lens..., vitreous, retina, choroid, sclera
\item organization of the retina: optic nerve head, macula, fovea, the difference layers of cells
\item how each layer is attached to each other
\item how incoming light is processed: travels through the retina, captured by photoreceptors, transformed in biochemical signal, transform in elecrtical signal, sent through the optic nerve to the brain
\item the vasculature in the retina and the choroid (add where it comes from, i.e., ophthalmic artery?)
\item ajouter des images de healthy/diseased retinae OCTA/FA and OCT
\item role du RPE
\end{itemize}

\large\textbf{organization of the eye: iris/lens..., vitreous, retina, choroid, sclera}

The retina is a layer of tissue, aproximately \SI{0.5}{\mm} thick, that lines the back of the eye.
The retinal inner surface is attached to the vitreous body, the clear gel that separates the lens and the retina, while its outer surface is attached to Bruch's membrane, see Figure~\ref{fig:architecture-eye}.
When looking at a cross section of the retina as shown in Figure~\ref{fig:architecture-eye}, eight layers can be identified, the innermost (closest to the vitreous) being the nerve fiber layer and the outermost the retinal pigmented epithelium (RPE).
In the centre of the retina is a oval-shaped pigmented area called the macula.

The macula is about \SI{5}{\mm} wide and is responsible for the highly detailed, central vision thanks to its high concentration of photoreceptor cells, namely rods and cones.
In particular, cones are highly concentrated in the centre of the macula, in a pit approximately \SI{1.5}{\mm} wide named the fovea.\\
Another important structure of the retina is the optic disc.
The optic disc appears as a round spot on \textit{en-face} views of the retina and is around \SI{1.8}{\mm} wide.
Through the optic disc passes the optic nerve, the fibers of which extend to form the nerve fiber layer, which transmits visual information to the brain.

Light hitting the retina travels through the retinal thickness to reach the photoreceptor layers, located just above the RPE.
Due to its pigmentation, the RPE acts as a buffer for remaining light beams, protecting the retina from light damage and preventing backreflection of light that may interfer with the visual outcome.
Activation of photoreceptors by photon starts a chain of biochemical reactions the transform light into a biochemical signal~\cite{Hurley2009}.
This biochemical signal is picked up by cells in the inner layers of the retina, which transform it into an electrical signal~\cite{Arslan2018}.
The electrical signal is then transmitted to the nerve fiber layer, which relays it to the brain via the optic nerve, enabling vision.

Visual functions require high metabolic activity from the retinal cells, which makes the retinal tissue very demanding in oxygen.
In fact, the rates of oxygen consumption per unit volume of tissue are comparable for the brain and the retina~\cite{Medrano1995}.
To sustain the demand in oxygen, the retina is equiped with a dense network of capillaries, that branches out of the central retinal artery (CRA) and finishes at the central retinal vein (CRV).
The CRA is a branch of the ophthalmic artery and the CRV drains into the superior ophthalmic vein.
Both the CRA and CRV enter and exit the retina along the optic nerve.
The CRA's branches spread across four plexi within the inner half of the retinal depth, namely, the superficial (SCP), intermediate (ICP), deep (DCP) and the radial peripapillary capillary plexus (RPCP).
The aggregation of adjacent plexi leads to the use of the terms superficial vascular complex (SVC) and deep vascular complex (DVC).

The inner retinal circulation provides oxygen for the inner \SIrange{60}{80}{\percent} of the tissue.
The perfusion of the remaining outer \SIrange{20}{40}{\percent} is covered by the choroid.
The choroid is a vascular layer of the eye that lies beinhd the RPE, from which it is separated by Bruch's membrane.
Bruch's membrane is a thin (\SIrange{2}{4}{\mu\meter}), permeable barrier that controls exchanges of gas, nutrients and byproducts of metabolic activity between the RPE and the choroid~\cite{Curcio2013}. 
The choroid is structured in three vascular layers.
From the outermost to the innermost: Haller's layer, Sattler's layer and the choriocapillaris (CC).
The diameter of the vessels in each layer decreases as it approaches the retina, with the CC only composed of capillaries.
The organization of the CC is atypical.
The capillaries of the CC are have wide lumen (the cavity delimited by the vessel's walls) and are arranged into a single plane, with many connection between adjacent capillaries, also known as anastomosis.
Furthermore, the capillaries' walls have opening (or fenestration), increasing their permeability~\cite{Torczynski1976}.\\
This particular architecture combined with a high blood flow allows the CC to provide enough oxygen through difusion across Bruch's membrane and the outer retina and clear wastes from the retina towards the systemic circulation.
In addition, the CC regulates the temperature of the macula~\cite{Parver1991}.




\begin{figure}[h]
  \centering
  \includegraphics[width=0.45\textwidth, height=7cm]{ArchitectureEye} % From Wikipedia retina
  \hfill
  \includegraphics[width=0.45\textwidth, height=7cm]{RetinaHistology}
  \caption{General architecture of the eye and the retina. This is for illustration, taken from Wikipedia and 'Glia and blood retinal barrier: effects of ocular hypertension' and rights need to be checked/the pics need to be eddited.}
  \label{fig:architecture-eye}
\end{figure}
\subsubsection{In disease}

\begin{itemize}
\item degradation of the blood supply chain (rarefaction des vaisseaux sanguins dans la retine (e.g. en cas de diabete) et le choriocapillaris (e.g. AMD)
\item augmentation de la pression occulaire (IOP) in, e.g., glaucoma
\item epaississement de la membrane de Bruch avec l'age et peut-etre AMD?
\item formation de drusen in AMD et avec l'age et le risque de detachement de la retine ou du RPE
\item neovascularisation et leakages
\end{itemize}

% \begin{spacing}{0.0}
\bibliographystyle{abbrvnat} % {ksfh_nat}
{\normalsize \bibliography{section_2}}
% \end{spacing}

\end{document}
