\documentclass[12pt,a4paper]{journal}
\usepackage[margin=1in]{geometry}

\usepackage{times}

\usepackage[utf8]{inputenc}
\usepackage[T1]{fontenc}
\usepackage{indentfirst}
\usepackage{amsmath}
\usepackage{amsfonts}
\usepackage{amssymb}
\usepackage{caption}
\usepackage{subcaption}
\usepackage{graphicx}
\usepackage{siunitx}
\sisetup{range-phrase=--, range-units=single}

\graphicspath{{img/}}
\usepackage[numbers, super, sort&compress]{natbib}


\usepackage{tcolorbox}

\begin{document}

\subsection*{Intravitreal injections}

\textit{In silico} models of the pharmacokinetics and pharmacodynamics of anti-VEGF molecules, mainly administered inside the vitreous, have been discussed in Section 5.
Here, we focus more specifically on models aimed at understanding and optimizing the surgical procedure of injecting drugs in the vitreous humor with a needle and how the outcome may be affected by the vitreous humour's properties.





\subsection*{Implants/port-delivery}


\subsection*{Subretinal and systemic administration}

The retina and the eye in general are thought to be isolated from systemically injected drugs by a number of biological barriers, including the choroidal circulation, the blood-retinal barrier, lymphatic clearance and the sclera.
Therefore, the systemic, or intravenous, administration of ocular drugs is uncommon.
However, understanding of the pharmacokinetics of intravenously injected drugs remains of interest since harmful effects on the retina have been reported.~\cite{Fu_2017}

In fact, significant ocular exposure, determined by a noncompartmental model, to intravenously injected antibodies has been reported.~\cite{Shivva_2021}
The possibility to deduct vitreous concentrations after intravenous injection has been demonstrated by a compartmental model.~\cite{Vellonen_2015}
However, the transfer rate from systemic circulation to the eye compartment was taken from a previous model and therefore represents the rate of clearance from the eye compartment into the systemic circulation.
The same model applied to experimental data of the permeability of the outer blood-retinal barrier found assymetric exchange rates between the choroid and the vitreous.~\cite{Ramsay_2019}
Additionally, the analysis suggests that the RPE may not be the main route of clearance from the vitreous for drugs entering the retina by the systemic circulation.~\cite{Ramsay_2019}


In contrast, injections into the subretinal space, has gained traction, in particular for the delivery of cell or gene therapy.
Subretinal injections provide direct access to the targeted cells, namely the photoreceptor and RPE cells.
In both cases, the blood-retinal barrier plays an important role in the total exposure of the tissue to the therapeutics.
Yet, the mechanisms of exchange between systemic circulation and the retina remain elusive and few \textit{in silico} models of subretinal or systemic administration of ocular drugs have been developed.





Some models have computationally looked at the unexposure of the eye to intravenously injected drugs.~\cite{Shivva_2021,Vellonen_2015}
Retinal exposure in the retina and 


\bibliographystyle{abbrvnat}
\bibliography{section-7}

\end{document}