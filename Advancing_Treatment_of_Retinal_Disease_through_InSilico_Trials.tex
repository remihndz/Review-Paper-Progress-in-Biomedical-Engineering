\documentclass{article}

\usepackage[subpreambles=false]{standalone}
\usepackage{import}

\usepackage[utf8]{inputenc}
\usepackage[T1]{fontenc}
\usepackage{amsmath}
\usepackage{amsfonts}
\usepackage{amssymb}
\usepackage{graphicx}
\usepackage{authblk}
\usepackage{fullpage}
\usepackage{subfigure}
\usepackage{color}
\usepackage{hyperref}
\usepackage{mathtools}
\usepackage{threeparttable}
\usepackage[figuresright]{rotating}

% \usepackage{mathptmx}
% \usepackage{mathrsfs}
% \usepackage{url}
\usepackage[super,numbers]{natbib}
\usepackage{xltabular}
\usepackage{tcolorbox}

\usepackage{siunitx}
\sisetup{range-phrase=--, range-units=single}

\graphicspath{{img/}}



\title{Advancing Treatment of Retinal Disease throuh \textit{in silico} Trials}

% \author[1,2]{R\'emi Hernandez}
% \author[3]{Paul A. Roberts}
% \author[1,2]{Wahbi K. El-Bouri}
% \affil[1]{Liverpool Centre for Cardiovascular Science, University of Liverpool and Liverpool Heart \& Chest Hospital Liverpool, UK}
% \affil[2]{Department of Cardiovascular and Metabolic Medicine, University of Liverpool, UK}
% \affil[3]{Centre for Systems Modelling and Quantitative Biomedicine, University of Birmingham, UK}
% \date{}


\author[1,2]{R\'{e}mi Hernandez\footnote{E-mail address: remi.hernandez@liverpool.ac.uk (R\'{e}mi Hernandez)}}                                                                                                     
\author[3]{Paul A. Roberts\footnote{E-mail address: p.a.roberts@univ.oxon.org (Paul A. Roberts)}}                                                                                                                  
\author[1,2]{Wahbi K. El-Bouri\footnote{E-mail address: w.el-bouri@liverpool.ac.uk (Wahbi K. El-Bouri)}}                                                                                                           
\affil[1]{Liverpool Centre for Cardiovascular Science, University of Liverpool and Liverpool Heart \& Chest Hospital Liverpool, UK}                                                                                
\affil[2]{Department of Cardiovascular and Metabolic Medicine, University of Liverpool, UK}                                                                                                                        
\affil[3]{Centre for Systems Modelling and Quantitative Biomedicine, University of Birmingham, Institute of Biomedical Research, Birmingham, B15 2TT, UK}                                                          
%                                                                                                                                                                                                                  
\renewcommand\Authands{ and }                                                                                                                                                                                      
%                                                                                                                                                                                                                  
%                                       
\begin{document}

\date{\vspace{-5ex}}
\maketitle

\section*{Introduction}

\begin{xltabular}{\textwidth}{ll}
  \toprule
  \caption{List of abbreviations.}
  \hline
  \textbf{Term} & \textbf{Definition}                                       \\
  \hline
  \multicolumn{2}{l}{\textit{\textbf{Mathematical terms}}}                  \\
    1/2/3D        & 1/2/3 spatial   dimensions                                \\
    DDE           & Delay differential   equation                             \\
    FEM           & Finite element method                                     \\
    ISCT          & \textit{In silico clinical trial}                         \\
    ODE           & Ordinary differential   equation                          \\
    PD/PK         & Pharmacodynamics/Pharmacokinetics                         \\
    (RD-)PDE      & (Reaction-diffusion)   partial differential equation      \\
    VP            & Virtual population                                        \\
    VVUQ          & Validation,   verification and uncertainty quantification \\
    \multicolumn{2}{l}{\textit{\textbf{Biological terms}}}                    \\
    AMD           & Age-related macular   degeneration                        \\
    ACoA          & Acetyl coenzyme A                                         \\
    AOSLO         & Adaptive optics   scanning laser ophthalmoscopy           \\
    BC            & Boundary condition                                        \\
    BrM           & Bruch’s membrane                                          \\
    BSG1          & Basigin-1                                                 \\
    CC            & Choriocapillaris                                          \\
    CFP           & Colour fundus   photography                               \\
    Ch            & Cholesterol                                               \\
    CIT           & Citrate                                                   \\
    CNV           & Choroidal   neovascularisation                            \\
    CORD          & Cone-rod dystrophy                                        \\
    CRA/CRV       & Central retinal   artery/central retinal vein             \\
    DA            & Dark adaptation                                           \\
    DCP           & Deep capillary plexus                                     \\
    DHA           & Docosahexaenoic acid                                      \\
    DR            & Diabetic retinopathy                                      \\
    DRC           & Deep retinal   capillaries                                \\
    DVC           & Deep vascular complex                                     \\
    (P/mf)ERG     & (Pattern/multi-focal)   electroretinogram                 \\
    F16BP         & Fructose-1,6-bisphosphate                                 \\
    FAF           & Fundus   autofluorescence                                 \\
    g             & Glucose                                                   \\
    G3P           & Glycerol-3-phosphate                                      \\
    G6P           & Glucose-6-phosphate                                       \\
    GA            & Geographic   atrophy                                      \\
    GLUT1         & Glucose transporter 1                                     \\
    ICP           & Intermediate   capillary plexus                           \\
    ILM           & Inner limiting   membrane                                 \\
    IPL           & Inner plexiform layer                                     \\
    IR            & Infrared reflectance                                      \\
    IRD           & Inherited retinal   disease                               \\
    IS            & Inner segment(s)                                          \\
    IVI           & Intravitreal   injection                                  \\
    LA            & Light adaptation                                          \\
    LACT          & Lactate                                                   \\
    MANF          & Mesencephalic   astrocyte-derived neurotrophic factor     \\
    mTOR(C1)      & Mammalian target of   rapamycin (complex 1)               \\
    NADPH         & Nicotinamide adenine   dinucleotide phosphate             \\
    nAMD          & Neovascular AMD                                           \\
    Ngb           & Neuroglobin                                               \\
    nnAMD         & Non-neovascular AMD                                       \\
    (SD-)OCT      & (Spectral-domain)   optical coherence tomography          \\
    OCTA          & Optical coherence   tomography angiography                \\
    ONL           & Outer nuclear layer                                       \\
    OPL           & Outer   plexiform layer                                   \\
    OS            & Outer segment(s)                                          \\
    PACF          & Photoreceptor   apoptosis commitment factor               \\
    Ph            & Photoreceptor(s)                                          \\
    PO$_2$        & Partial pressure of   oxygen                              \\
    PYR           & Pyruvate                                                  \\
    RCD           & Rod-cone dystrophy                                        \\
    RdCVF(1/2)(L) & Rod-derived cone   viability factor(1/2)(-long)           \\
    ROS           & Reactive oxygen   species                                 \\
    RP            & Retinitis   pigmentosa                                    \\
    RPE           & Retinal pigment   epithelium                              \\
    RPCP          & Radial peripapillary   capillary plexus                   \\
    SVC           & Superficial vascular   complex                            \\
    SVP           & Superficial vascular   plexus                             \\
    VEGF          & Vascular endothelial   growth factor \\
    \hline
  \end{xltabular}
  
The highly detailed images our eyes are capable of capturing necessit the close interplay of many different cells and structures.
Degradation of visual functions can severely affect the quality of life of patients.
The retina is a complex and fragile tissue playing a major role in visual functions.
As such, it is perhaps not surprising that many severe visual impairments find their root in the retina.

A number of devices exist to observe the retina and clincially relevant biomarkers, often noninvasively.
For instance, optical coherence tomography (OCT) and its angiographic extension enable three dimensional, high resolution scans of the retinal structure and vessels in a matter of seconds and noninvasively.
These devices undoubtedly improve clinical care by providing accurate measurements of biomarkers, such oedema.
Still, treatment strategies tend to be standardized, following the guidelines of successful clinical trials.
However, the treatment response may vary between patient, indicate that alternative or tailored treatments may be needed.
The development of new treatment is long, arduous and expensive, with only a small fraction of clinical trials succeeding, mostly due to an inability to prove effectiveness\cite{Fogel_2018}.

\textit{In silico} modelling can offer insights on the underlying causes of disease, which are often hard or impossibe to obtain by experimental or observational means.
The past two decades have seen a rise in the use of such models for basic research in biology and medicine.
Furthermore, the use of digital evidence to inform clinical trials is also gaining traction. 
The recent increase in the number of literature reviews on mathematical models of the eye and the retina and the development of platforms for simulating the eye accessible to non-modellers are evidence of that interest\cite{Arciero_2019,Arciero_2017,Bhandari_2021,Harris_2013,Prudhomme_2021,Roberts_2016,Sala_2018}.

\textit{In silico} trials can enhance both clinical care and traditional clinical trials, preserving sight for the fast growing number of patients suffering from retinal disorders.
Running virtual clinical trials requires a close collaboration between experimentalist, modellers and clinicians (see Figure~\ref{fig:ModellingCycle}).
Other medical discplines such as oncology and cardiology have already benefitted from \textit{in silico} trials\cite{Gaffney2022,Ravvaz2017}.
Ophthalmology can also benefit from computer aided clinical development.
Indeed, the retina provides researchers with a wealth of images and measurements that is perhaps unparalleled in other medical specialties.

In this paper, we will review the state-of-the-art model of the retina, both in health and disease, and highlight the challenges to developing \textit{in silico} clinical trials for retinal diseases.
The remainder of this paper is organized as follows.
In Section 2, we provide a brief introduction to the physiology of the retina, its main features and summarize major retinal pathologies and available treatments.
In Section 3, we define terms and concepts related to \textit{in silico} modelling.
Sections 4 and 5 review models of the retinal physiology in health and disease and models of the treatment of major retinal diseases.
Section 6 is dedicated to models of therapeutic procedures.
Section 7 defines \textit{in silico} trials and provide the reader with examples of their successful application.
Finally, in Section 8, we summarize the state of mathematical models of the retina and provide a plan of action to achieve \textit{in silico} clinical trials for retinal pathologies.

\section*{Retinal physiology and therapeutic approaches}

\subsection*{Physiology}

\subsubsection*{In health}

\paragraph*{Retinal Layers and Cells and the visual cycle.}

The retina is a layer of tissue less than \SI{0.5}{\mm} thick, that lines the back of the eye.~\cite{Gupta_2015}
It is broadly split into two layers: the retinal inner surface is attached to the vitreous body, the clear gel that separates the lens and the retina; while its outer surface is attached to Bruch's membrane (Figure~\ref{fig:architecture-eye}).
When looking at a cross section of the retina as shown in Figure~\ref{fig:architecture-eye}, several sub-layers can be identified. The innermost (closest to the vitreous) being the nerve fiber layer and the outermost being the retinal pigmented epithelium (RPE).
Each layer of the retina contains a different type of cell.
The layers located above the RPE are made of neurons and synapses and is referred to as the neurosensory retina.
In the centre of the retina is an oval-shaped pigmented area called the macula that can be seen on photographs of the retina (Figure~\ref{fig:Scans}).
The macula is about \SI{5}{\mm} wide and is responsible for the highly detailed, central vision, owing to its high concentration of photoreceptor cells, namely rods and cones.
In particular, cones are highly concentrated in the centre of the macula, in a pit approximately \SI{1.5}{\mm} wide named the fovea.
The fovea is a critical area of the retina, as this is where visual acuity is highest.
It is also the area where a number of pathological conditions present.

\begin{figure}[t!]
  \centering
  \includegraphics[width=0.45\textwidth, height=7cm]{ArchitectureEye} % From Wikipedia retina
  \hfill
  \includegraphics[width=0.5\textwidth, height=7cm]{RetinalLayers}
  \caption{Anatomy of the eye and the retina. \textbf{(Left)} Anatomy of the eye with the retina identified. Image credits: Rhcastilhos and Jmarchn, CC BY-SA 3.0, via Wikimedia Commons. \textbf{(Right)} The retina and its subdivision into layers. From inner retina to outer retina: inner limiting membrane (ILM), nerve fiber layer (NFL), ganglion cell layer (GCL), inner plexiform layer (IPL), inner nuclear layer (INL), outer plexiform layer (OPL), outer nuclear layer (ONL), photoreceptors (PHR), retinal pigmented epithelium (RPE), Bruch's membrane (BrM), choriocapillaris (CC), Sattler's layer (SL), Haler's layer (HL). Modified from the work of Trivi\~no et al., published under CC BY 3.0 license.\cite{Trivino_2012}}
  \label{fig:architecture-eye}
\end{figure}

\begin{figure}[t!]
  \centering
  \includegraphics[width=0.45\textwidth]{FA}
  \caption{Anatomical features of the healthy retina on a fundus photograph. By Mikael H\"aggstr\"om, used with permission.}
  \label{fig:Scans}
\end{figure}

% Another important structure of the retina in disease is the optic disc.
% The optic nerve is an essential piece of the visual cycle as it allows transfer of electric signals produced by phototransduction.
% The optic disc appears as a round spot on fundus photographs of the retina (Figure~\ref{fig:Scans}) and is around \SI{1.8}{\mm} wide.
% Through the optic disc passes the optic nerve, the fibers of which extend to form the nerve fiber layer, which transmits visual information to the brain.
% The optic nerve is composed of the axons of the retinal ganglion cells, amounting to between $500,000$ and $1.2$ million, and is surrounded by connective tissue.~\cite{Salazar_2019}

Phototransduction, the conversion of light into an electrical signal, results from a cascade of events in the retina. 
Visual information originates from light hitting the retina.
The light crosses the retinal thickness to reach the body of the photoreceptors, in the layer just before the RPE (see Figure~\ref{fig:architecture-eye}).
Activation of photoreceptors by photons starts a cascade of biochemical reactions that transforms light into a biochemical signal.~\cite{Hurley_2009}
This biochemical signal is picked up by cells in the inner layers of the retina, which transform it into an electrical signal.~\cite{Arslan_2018}
The electrical signal is then transmitted to the nerve fiber layer, which relays it to the brain via the optic nerve, enabling vision.
The optic nerve is composed of the axons of the retinal ganglion cells, amounting to between $500,000$ and $1.2$ million, and is surround by connective tissue\cite{Salazar_2019}.
The optic nerve enters the retina through the optic disc, seen as a round spot, around \SI{1.8}{\mm} wide, on fundus photographs of the retina (Figure~\ref{fig:Scans}).
The optic disc is an important structure of the retina in disease and is closely monitored by clinicians for symptoms of, e.g., glaucoma.

Owing to its pigmentation, the RPE acts as a buffer for any remaining light, protecting the retina from light damage and preventing backreflection of light that may interfere with the visual outcome.
The RPE is formed of a single layer of epithelial cells, whose primary function is to support photoreceptors.
To fulfil this purpose, the RPE acts as the blood-retinal barrier, regulating the transport of ions, fluid, proteins and other molecules.~\cite{Boulton_2001}

The photoreceptors are attached to the RPE cells' plasma membrane.
Through this close interaction, the RPE collects by-products of the photoreceptors, a washout essential to maintain a healthy neurosensory retina.
The RPE also transports nutrients such as retinoids to the cell where they are essential for the production of the light-sensitive rhodospin protein.~\cite{Boulton_2001} 
Those by-products are released into the systemic circulation via the choroid, one of the two circulations of the retina.~\cite{Boulton_2001}
The cells of the RPE also play a major role in the immune response in the retina and in maintaining the choroidal vasculature through secretion of various vascular growth factors.~\cite{Boulton_2001,Detrick_2020} 

\paragraph*{Blood Supply to the Retina}

Visual functions require high metabolic activity from the retinal cells, which makes the retinal tissue very demanding in oxygen.
In fact, the rates of oxygen consumption per unit volume of tissue are comparable for the brain and the retina.~\cite{Medrano_1995}
To sustain the demand in oxygen, the retina is equipped with a dense network of capillaries, that branches out of the central retinal artery (CRA) and finishes at the central retinal vein (CRV).
The CRA is a branch of the ophthalmic artery and the CRV drains into the superior ophthalmic vein.
Both the CRA and CRV enter and exit the retina along the optic nerve.
The CRA's branches spread across four plexi within the inner half of the retinal depth, namely, the superficial vascular plexus (SVP) and the intermediate (ICP), deep (DCP) and the radial peripapillary capillary plexus (RPCP).
The terms superficial vascular complex and deep vascular complex are sometimes used to designate the RPCP-SVP complex and the ICP-DCP complex, respectively.
The inner retinal circulation provides oxygen for the inner \SIrange{60}{80}{\percent} of the tissue.
The perfusion of the remaining outer \SIrange{20}{40}{\percent} is covered by the choroid.

The choroid is a vascular tissue of the eye that lies behind the RPE, from which it is separated by Bruch's membrane.
Bruch's membrane is a thin (\SIrange{2}{4}{\micro\meter}), permeable barrier that provides structural support to the RPE and regulates gas and mass exchanges with the choroid.~\cite{Curcio_2013}
The choroid is structured in three vascular layers.
From the outermost to the innermost: Haller's layer, Sattler's layer and the choriocapillaris (CC).
The diameter of the vessels in each layer decreases as it approaches the retina, with the CC only composed of capillaries.
The vascular input to the choroid is provided by the short posterior ciliary arteries (between 6 and 12), which are branches of the ophthalmic artery.~\cite{Kiel_2010}
The large vessels of the two outer layers of the choroid run parallel to the retinal axis.
Some arterioles from Sattler's layer branch at an almost \SI{90}{\degree} angle to perfuse the CC.~\cite{Nickla_2010}
Each of these arterioles form an hexagonal-shaped domain, fed by a single arteriole and drained by a varying number of venules back into Sattler's layer.~\cite{Zouache_2016}
The capillaries of the CC have wide lumen (the cavity delimited by the vessel's walls), between \SIrange{7}{40}{\micro\meter} in the CC against \SIrange{5}{10}{\micro\meter} in the retina, and are arranged into a single plane, with many connections between adjacent capillaries, also known as anastomosis.~\cite{Bill_1983, ChanLing_2011,Fryczkowski_1994}
Furthermore, the capillary walls have openings (or fenestrations), increasing their permeability.~\cite{Nickla_2010}
The fenestrations on the capillary walls are more numerous on the side facing the retina and are at least wide enough to let molecules with a diffusional radius of \SI{3.7}{\micro\meter} pass into the blood.~\cite{Nickla_2010, Bill_1983}
This particular architecture combined with a high blood flow allows the CC to provide enough oxygen through diffusion across Bruch's membrane and the RPE and to clear waste from the retina through the systemic circulation.

The blood supply from the choroid to the outer retina is indirect.
Therefore, the choroid has high blood flow and a high oxygen content.~\cite{Bill_1983}
The high concentration of oxygen in the choroidal circulation creates a strong gradient for its diffusion between the CC and the outer retina.
The high blood flow rates may also help regulating the temperature of the macula, by keeping it at the same temperature as the rest of the body.~\cite{Bill_1983, Parver_1991}

While appropriate blood supply is necessary to maintain vision, blood vessels can interfere with light and hinder resulting images.
The distribution of photoreceptors in the retina is heterogeneous, with a higher concentration in the parafovea, for rods, and fovea, for cones.~\cite{Zouache_2022}
For this reason, the centre of the fovea is an avascular zone around \SI{500}{\micro\meter} wide, often referred to as the foveolar avascular zone.

The interactions between layers of the retina, e.g., the RPE and photoreceptors, is essential for visual function.
Disruption of the delicate structure of the retina leads to degeneration of sight.

\subsubsection*{In disease}

While the retina is a key element for visual function, its relative fragility makes it susceptible to a number of conditions, some of which may lead to blindness.
Loss of visual acuity can be caused by failures at any level of the phototransduction cascade.\\

\paragraph*{Dry age-related macular degeneration.}
The dry form of age-related macular degeneration (AMD) is a common condition in individuals over fifty years of age.
It is characterised by the presence of accumulated cellular debris under (drusen) and above (reticular pseudo-drusen) the RPE.~\cite{Bottos_2012}
The drusen and pseudo-drusen appear when accumulated stress from age and other environmental factors such as cigarette smoking and diet cause the dysfunction of the RPE's clearance functions.
In the end-stage of dry AMD, RPE cells degenerate and form geographic atrophy in the macula.~\cite{Jager_2008}
Geographic atrophy is the name given to patches of atrophied RPE cells that can be seen on fundus photographs.
Ultimately sight loss is due to the atrophy of the photoreceptors deprived from the support of underlying RPE cells, but a gradual distortion of images is also present from the early stages.~\cite{Newsom_2008,Zacks_2022}\\

\paragraph*{Retinitis pigmentosa.}
Similarly, individuals with retinitis pigmentosa (PR) progressively lose sight from degeneration of the photoreceptors.
Disease-related mutations are typically expressed by rods, leading to their degeneration in early stages of PR.
For a reason still unclear, cones cell degeneration follows from the absence of rods, despite cones not expressing the mutations responsible for rod death.
Multiple hypothesis have been suggested for this phenomenon.
One of them is the oxygen toxicity hypothesis.
As the choroid lacks the capacity to regulate blood flow accordingly to metabolic needs of the retina, some suggested that the absence of rods leads to high oxygen tension in the photoreceptor layers, an hyperoxia detrimental to cones survival.~\cite{Roberts_2018,Stone_1999}
Another hypothesis speculates that cones survival is dependent on a trophic factor provided by rods.
Therefore, the later degeneration of cones could be explain by the absence of such factors.~\cite{Roberts_2022}\\

\paragraph*{Diabetic retinopathy.}
One of the consequences of diabetes is the degeneration of the smaller vessels in the retina, the capillaries.
Such degeneration include an increase in vascular permeability, a loss of the pericytes coating capillary walls and a thickening of the endothelial basement membrane upon which endothelial cells are attached.~\cite{Medina_2016}
Additionally, a rarefaction of capillaries can be observed around the foveal avascular zone.
Such microvascular degeneration may cause microvascular occlusions, haemorrhages and oedemas as well as macular ischaemia.~\cite{Medina_2016}
The subsequent release of vascular endothelial growth factor (VEGF) may progress the diabetic retinopathy (DR) to its proliferative stage.
In proliferative DR, elevated VEGF levels cause the growth of abnormal capillary membranes in a process known as neovascularization.
The neovascular membranes cause further haemorrhages and oedemas and scaring of the tissue.
Scaring and the toxicity of blood damage the photoreceptors, sometimes permanently, resulting in dramatic loss of sight.~\cite{Friedlander_2007,Gupta_2015}
Moreover, the neovasculature may pierce through the inner limiting membrane (ILM) and haemorrhage in the vitreous or preretinal space, the potential space between the ILM and the vitreous, disturbing vision.\\

\paragraph*{Neovascular age-related macular degeneration}
The neovascular form of AMD is characterised by a much more dramatic loss of sight compared to its dry form and the appearance of macular neovasculature.
As most cases of neovascular AMD happen in eyes affected by dry AMD, it may be referred to as a late stage of AMD.
The current understanding of the pathogenesis of neovascular AMD is that, in addition to other age-related changes in the retina, the progressive buildup of materials in Bruch's membrane characteristic of dry AMD induces hypoxic conditions in the outer retina.~\cite{Jager_2008,Newsom_2008}
The subsequent upregulation of VEGF by RPE cells stimulate angiogenesis in the CC.~\cite{Jager_2008}
The resulting neovasculature may pierce through Bruch's membrane, growing below the RPE, or through the RPE, growing in between the RPE and the photoreceptors, in the subretinal space.
The neovasculature may also emerge from the inner retinal circulation.
Regardless of the location, neovascular membranes are associated with oedema, haemorrhage, scarring and RPE or retinal detachment (when the retina separates from the RPE), all of which cause the quick loss of sight observed in neovascular AMD.~\cite{Gupta_2015,Jager_2008}\\

\paragraph*{Retinal tears and breaks}
Tears may occur in the retina for various reasons, including trauma and surgery.
Posterior vitreous detachment is another reason for the apparition of tears and breaks (tears through the whole thickness of the neurosensory retina).
With age, the vitreous humor undergoes progressive liquefaction.
In addition, there is gradual loss of adhesion between the ILM and the posterior vitreous cortex to which it is attached.~\cite{Bottos_2012,Medina_2016}.
The combination of those two factors facilitates the accumulation of vitreous fluid in the preretinal space, tearing apart the ILM and the vitreous.
However, in areas of strong adhesion of the ILM, the mechanical forces induced by the fluids can cause retinal tears or breaks and macular holes.~\cite{Shechtman_2009}
Macular holes cause blurry vision while retinal tears provide a path to the subretinal space which may fill with fluids, causing retinal detachment.~\cite{Medina_2016}
Additionally, the traction caused by the fluid in the preretinal space can cause retinal detachment.\\

\paragraph*{Retinal vessel occlusion}
A number of conditions may cause blood vessels in the retina to clog or collapse.
Accumulation of plaque caused by, for example, cholesterol, or a blood clot (embolus) in blood vessels form an obstruction to blood flow.~\cite{Medina_2016}
Sickle cell retinopathy causes a stiffening of red blood cells which can result in occlusion of arterioles or capillaries, but also of the CRA.~\cite{Medina_2016}
Mechanical pressures such as ocular hypertension can cause the collapse of veins, including the CRV, when the external pressure becomes higher than the blood pressure.~\cite{Hayreh_2004}
Stiffening of the CRA is suspected to also compress the CRV.~\cite{Medina_2016}
Occlusion of arteries causes non-perfusion areas in the retina, stopping visual functions in the affected areas.
Irreversible damage to the retina starts appearing after \SI{100}{\min} of non-perfusion.~\cite{Hayreh_2004}
Some patients with retinal vein occlusion may develop retinal ischaemia, though the reasons are unclear.~\cite{Khayat_2018}
Other symptoms include retinal oedema and, in some cases, neovascularization in various locations, including the optic disc and the retina.~\cite{Medina_2016}
\\

\paragraph*{Glaucoma}
Glaucoma is a group of optic neuropathies affecting the optic nerve and the ganglion cell layer of the retina (see Figure~\ref{fig:architecture-eye}).
Ophthalmologists observe a thinning of the ganglion cell layer and degeneration of the tissue of the optic nerve.
Those symptoms are due to degeneration of the ganglion cell bodies (in the ganglion cell layer) and axons (forming the optic nerve).~\cite{Quigley_2011}
it is thought that a combination of ocular hypertension and changes in cerebrospinal fluid pressure and retinal blood pressure create a pressure gradient around the optic nerve, increasing mechanical stresses on the tissue.~\cite{Band_2009,Nickells_2012}
Vision loss in glaucoma is due to the loss of connectivity between the retina and the brain and cannot be recovered.~\cite{Quigley_2011} 


\subsection*{Therapeutic strategies}

\subsubsection*{Surgeries}

Retinal detachment often requires emergency surgery to preserve vision.
Different methods are available to reconnect the retina with the RPE.
One possibility is to insert a so-called scleral buckle in the sclera, the white tissue supporting the eyeball and located behind the choroid in the retina.
The scleral buckle will push the sclera towards the detached retina.~\cite{Sodhi_2008}
Another approach consists of pushing the retina towards the RPE.
This technique, called pneumatic retinopexi, consists of injecting a gas bubble into the vitreous~\cite{Sodhi_2008}.
The tension created by the gas pushes the retina back into contact with the RPE.
Once contact is restored, the RPE can drain the liquid that accumulated in the retinal holes.
The gas is naturally removed from the vitreous.

In more complicated cases of retinal detachment, e.g., in the presence of vitreous haemorrhage, vitrectomy might be preferred to pneumatic retinopexi.
Vitrectomy consists of replacing the vitreous humour with either a silicone oil substitute or a gas bubble, favouring reattachment in a similar fashion to pneumatic retinopexi.~\cite{Dervenis_2022}
Surgeries for retinal detachment may also be complemented with ILM peeling, cryotherapy or laser photocoagulation.
Because the ILM contributes to retinal rigidity and in vitreal traction of the retina, its removal may be useful to facilitate closure of retinal holes and lower the rate of reopening after surgery.~\cite{Chatziralli_2018}
Cryotherapy and photocoagulation can be performed to scar the tissue around retinal tears, effectively sealing them to prevent spread that may lead to retinal detachment.
Cryotherapy and photocoagulation can be used alone or after pneumatic retinopexi or vitrectomy.~\cite{Sodhi_2008}

Photocoagulation may also be used to address abnormal neovasculature in DR or vessel occlusion.~\cite{Evans_2014}
The burn induced by the laser therapy to the retinal tissue is aimed to stop further vascular growth by either sealing or ablating the aberrant blood vessels (focal laser photocoagulation) or by ablating in a wider range (panretinal laser treatment).
In the case of panretinal laser treatment, it is thought that the damage brought to the retinal tissue causes a change in the oxygen supply and demand balance that could lower VEGF expression.~\cite{Evans_2014}
Photodynamic therapy is a similar approach which uses photochemical mechanisms to induce cell death rather than heat which may be preferred for destroying neovasculature in the more fragile foveal region, for example in eyes with neovascular age-related macular degeneration.~\cite{SchmidtErfurth_2000}

In eyes with glaucoma, the aim of surgery is to lower intraocular pressure.
This can be done by decreasing the pressure in the aqueous chamber.~\cite{Quigley_2011}
Laser treatment of the trabecular meshwork, an area of tissue around the base of the cornea (Figure~\ref{fig:architecture-eye}), is aimed to improve drainage of the aqueous humor.
In some cases, parts of the trabecular meshwork may be removed to allow further drainage in a procedure called trabeculectomy.
Conversely, intraocular pressure can be lowered by decreasing the production of aqueous humor, acting on the inflow rather than the outflow of aqueous humor.
This treatment, referred to as cyclodiode therapy, aims to reduce this inflow by destroying the ciliary body that produces aqueous humor.~\cite{Allbon_2022}

\subsubsection*{Drug based}

The regulation of intraocular pressure can be achieve with drugs.
In fact most patients with glaucoma begin treatment with eye-drops, used daily to reduce intraocular pressure.
The eye-drops are designed to lower the aqueous humor production, increase drainage of aqueous humor, or both.~\cite{Chakrabarti_2022,Quigley_2011}

Retinal vascular disorders such as neovascular age-related macular degeneration, proliferative DR or retinal vein occlusion, are for the most part treated with anti-VEGF.~\cite{Andreoli_2007,Kim_2021}
These drugs, mostly injected in the vitreous humor, bind to VEGF molecules that drive the growth of immature and leaky blood vessels.
Anti-VEGF treatment has proven effective to suppress disease progression, reducing retinal oedema and restoring vision.~\cite{Andreoli_2007,Heier_2006,Kim_2021}
Steroids, either oral or intravitreal, can also be prescribed to reduce the production of pro-inflammatory cytokines such as VEGF in order to increase the RPE functions in patients with RP presenting macula oedema.~\cite{Strong_2016}

Deterioration of visual functions due to ageing (cellular senescence), whether normal or fasten by disease, has been addressed by a range of therapeutics called senotherapeutics.
Their use for the is being investigated as a mean to maintain vision in disease such non-neovascular age-related macular degeneration which lack therapeutic targets.~\cite{Lee_2021}


\subsubsection*{Other}

Alternative intervention types are possible, however, their use remain rare as they are investigated for efficiency and safety.
Retinal prosthesis have started being approved for use in late stages of certain retinal diseases such as RP.~\cite{Luo_2016}
The artificial retina replaces the degenerated photoreceptors by a video camera.~\cite{Luo_2016,Stingl_2017}
The pixelated image is then transferred to the inner retina's neurons, relatively spared by RP, using microelectrodes.
The transfer of the electrical signal to the brain is then achieved normally through the optic nerve.~\cite{Luo_2016,Stingl_2017}

Because matured RPE and photoreceptor cells are unable to divide and proliferate, stem cell therapy is a promising therapeutic approaches in eyes with photoreceptor or RPE atrophy such as age-related macular degeneration.~\cite{Berta_2011,Stern_2015}
Stem cells are naturally found in the body and have yet to form into a specific cell type hence have the ability to form into many different cell types.
Therefore, stem cells injected into the retina are able to replace degenerated RPE or photoreceptor cells and halt loss of vision or even improve vision.~\cite{ONeill_2020}

Gene therapy is an intervention that aims to slow the progress of inherited retinal diseases such as RP.
It consists of introducing copies of healthy genes into the retina in order to reduce degeneration from disease-related mutations.~\cite{Battu_2022}
Viruses are designed to safely deliver the healthy genes to cells, which will either replace or silence the mutant genes.~\cite{Battu_2022} 









\section*{Mathematical modelling}

Broadly speaking, a mathematical or \textit{in silico} model describes a biological system by a number of independent and dependent variables and a set of equations or rules relating them that govern the state of the system.
When designing an \textit{in silico} model, one has to make choices among the variety of model types available.
This choice is informed by the system to be modelled, the question to be answered and the resources available (e.g., data or computational resources).
Because biological systems are so complex, the mechanisms modelled need to be kept to a minimum in order for the simulations to be tractable and interpretable.
The process of designing and refining an \textit{in silico} model follows a logic similar to experimental sciences and is summarised in Figure~\ref{fig:ModellingCycle}.
In this section, we give a brief overview of different model types that can be used to model the retina and typical challenges emerging when modelling biological systems.
A more comprehensive overview can be found in the work of Roberts et al.\cite{Roberts_2016}

\begin{figure}[t!]
  \centering
  \includegraphics[width=.9\linewidth]{ModellingCycle}
  \caption{Modelling cycle of the design of an \textit{in silico} model. Reused with permission from Roberts et al.\cite{Roberts_2016}}
  \label{fig:ModellingCycle}
\end{figure}

\subsubsection*{Mechanistic vs phenomenological models}
\textit{A priori} understanding of the system can be used to inform model design, in which case the model is qualified as mechanistic.
If appropriate assumptions are made, the predictions of a mechanistic model will match observations.
Else, the \textit{a priori} assumptions may need to be reviewed and the model changed accordingly until satisfying agreement is found.
In contrast, phenomenological models simulate parts or the entirety of the system by arbitrary functions based on observation of the data.
For instance, an exponential decay is often assumed for the time course of drug in the body.
Phenomenological models are often used to simplify an overly complex or unknown mechanism acting on the system to be modelled.
It should be noted that simplifications are inherent to model development and, therefore, no model is fully mechanistic.

All models involve a number of parameters.
Some of those parameters may be known, either from direct measurements or theoretical formulae.
However, others may be unknown because they cannot be measured or are model-specific and may not represent a measurable, physical quantity.
In such cases, \textit{in silico} models can provide estimates of the physical quantity by fitting parameters to data, when available.
Otherwise, analysis of the model predictions may still provide useful insights on the system's behaviour.
Indeed, in either cases, tools such as sensitivity analysis and bifurcation analysis can provide insights on the system's behaviour for different parameter regimes.

\subsubsection*{Initial and boundary conditions}
While initial conditions are the state of the system at time $t=0$, boundary conditions define the behaviour of the system at the boundaries of a spatial domain.
Classical boundary conditions may set the value of the simulated function itself and is referred to as a Dirichlet boundary condition. The normal derivative along the boundary being set is called a Neumann boundary condition. A linear combination of the previous two types forms a Robin boundary condition.
Up to a few exceptions, initial and boundary conditions are necessary for differential equations to be well-posed and solvable.

\subsubsection*{Deterministic and stochastic models}

When random events are left out of a model, two simulations will yield the same results.
This kind of model is referred to as deterministic.
In opposition, stochastic models include a probabilistic component, e.g., the random movement of cells, which yields different solutions for each simulation.
Stochastic models are often used for discrete models e.g. agent-based models.
Discrete models simulate the behaviour of individual entities, such as cells, based on a set of arbitrary rules.
While discrete models can incorporate more details, continuum models allow for rigorous mathematical analysis.
A further limitation of discrete models includes the drastic increase in computation times as the number of objects simulated increase.

\subsubsection*{Differential equations}

Many models describe the spatial and temporal behaviour of a quantity within a biological system using differential equations.
Differential equations relate one or more unknown functions with their derivatives.
If the equation involves derivatives with respect to a single independent variable, it is referred to as an ordinary differential equation (ODE).
Otherwise, if more than one type of derivative appears in the equation, it is referred to as a partial differential equation (PDE).
Differential equations are mathematically tractable and can be solved analytically, when an algebraic expression of the unknown function is found, or numerically.

PDEs often arise when the spatial distribution of the quantity of interest is needed.
However, the spatial aspect is not always relevant to the research question being investigated.
In such cases, an ODE relating a function and its time derivative can be used.
ODEs can also be used to model a system along a single spatial dimension at steady state.

Assuming a steady state is a common simplifying assumption which can help reduce the complexity and increase the interpretability of the system by ridding the model of the time component.
This reduction follows from assuming that the system, when not perturbed, does not change over time, hence the time derivative of the functions describing it are null.
Similarly, spatial derivatives can be suppressed by assuming that the quantity is homogeneously distributed within the space.
This simplification, referred to in certain context as a well-mixed assumption, is often used to derive compartment models.

Compartmental models describe a system as separated compartments that can interact with each other, e.g., drug concentration in the eye can be described by a vitreal, retinal and choroidal compartment, with mass exchanges between adjacent compartments.
Likewise, lumped parameter models simplify systems with certain spatially distributed objects, such as blood flow within a network of capillaries.
In such models, the spatial aspect is reduced to a smaller number of entities to which the smaller objects are assigned, similarly to how capillaries are grouped into capillary plexi.
To each entity is attached a number of parameters which encompass the various properties of the spatially distributed system.
In the capillary blood flow example, these properties could include vascular resistance, oxygen extraction or storage capacity within each capillary of the network.
Both compartmental models and lumped parameters models describe the system with a set of ODEs for each of the compartments or entities.

Systems of ODEs are typically simpler to solve, both analytically and numerically, and to mathematically analyse compared to PDEs.
Challenges in the analysis of PDEs arise from the possibly complex geometry of biological systems.
Approximating the geometry with simpler shapes such as spheres and rectangles, may allow derivation of analytical solutions or, at least, greatly reduce the time needed to numerically solve the equations.

Numerically, differential equations are solved by discretising (that is, reducing the continuous variable to a finite number of points) the space and time variables into a mesh, at the corner of which the unknown's value is found by solving a matrix equation.
The size of the matrix depends on the refinement of the mesh, with smaller mesh sizes yielding larger matrices.
Therefore, mesh size provides some degree of control over the computational complexity of the model, that is the time and memory needed to run a simulation.

Still, systems with multiple or spatial scales, which often arise in biology, can be challenging to solve within reasonable times.
Indeed, smaller scales enforce smaller mesh sizes in order to account for the refined details of the geometry or of the mechanisms working on different time scales.
This issue appears, for example, when modelling the delivery of oxygen from retinal capillaries to a slab of tissue.
While the size of the tissue may be of the order of millimetres, the numerous blood vessels are, in contrast, of the order of micrometres, hence enforcing a mesh size of the order of micrometres.
In such problems where separate mechanisms interact with each other (e.g., oxygen transport in the capillaries and oxygen diffusion in the tissue), the unknown functions appear in more than one equation.
The model is then said to be coupled.
While coupling adds to the difficulty of solving the equations, clever mathematical techniques may help in reducing the complexity.
However, those tricks are often problem-specific and hard to generalise.

In addition, the often nonlinear behaviour of biological processes can affect the computational cost of running simulations.
Indeed, solving nonlinear equations most often requires to solve a series of subproblems until the difference between successive iterations becomes small enough.

When complex nonlinearities are combined with multiple scales, the time required to solve a set of equations increases dramatically.
For those reasons, computation time remains a barrier to the use of \textit{in silico} models for real-time simulations in the clinic.






\section*{Retinal haemodynamics, vascular diseases, neovascular age-related macular degeneration and diabetic retinopathy}

Adequate blood flow is essential for the supply of nutrients and removal of cellular wastes required to maintain visual functions.
The atypical dual circulation of the retina is both complex and fragile.
It is thought that the inner vessels perfuse the inner \SIrange{60}{80}{\percent} of the retina while the choroid supplies the remaining, more metabolically active, outer layers, around the photoreceptors.~\cite{Birol_2007}
It is also known that ocular blood flow is affected by, among other things, intraocular pressure, systemic blood pressure, metabolic activity and oxygen saturation of the blood.~\cite{Birol_2007,McCullough_1997,Palkovits_2014,Polska_2007,Pournaras_2008,Riva_1997,Wang_2014}
Many retinal disease have been linked, whether directly or indirectly, with abnormal haemodynamics.~\cite{Hayreh_2004,Medina_2016}
In fact, in certain diseases such as AMD and DR, the blood supply is so disrupted that pathological neovasculature starts invading the retina.
To prevent or halt such complications, clinicians often resort to anti-VEGF drugs.

In this section, we review \textit{in silico} models that contribute to our understanding of the normal physiology of the retina and models aiming at deciphering the aetiology of various diseases which may be linked with haemodynamics in the retina and the underlying choroid.
More comprehensive reviews of models of the microcirculation, angiogenesis and oxygen delivery can be found in the work of Arciero et al.~\cite{Arciero_2017, Arciero_2019}
In addition, we present models of the pharmacokinetics (PK) and pharmacodynamics (PD)


\subsection*{Retinal haemodynamics}

\begin{figure}[t!]
  \centering
  \includegraphics[width=0.45\textwidth, height=5.3cm]{cropped-RBC-in-capillaries.jpg}
  \hfill
  \includegraphics[width=0.45\textwidth, height=5.3cm]{EffectiveViscosity-Secomb.jpeg}
  \caption{Example of the effect of capillary calibre on the flow of blood. \textbf{Left}: Red blood cells in suspension in blood flowing within tubes of different diameter: \SI{4.5}{\micro\meter} (top), \SI{7}{\micro\meter} (middle) and \SI{15}{\micro\meter} (bottom). One can observe the flow of cells converging into a single file, surrounded by a layer of plasma as the diameter moves from \SI{15}{\micro\meter} to \SI{7}{\micro\meter}. The cell's shape adapt to fit into the tube as the diameter reaches the cell's size (top row). Reproduced with permission from Secomb.~\cite{Secomb_2003} \textbf{Right}: Example of an empirical law for the effective viscosity of blood accounting for the F\r{a}hr\ae us-Lindqvist effect and increased vascular resistance in smaller capillaries from the work of Secomb and Pries~\cite{Secomb_2013}.}
  \label{fig:effectiveViscosity}
\end{figure}

Retinal haemodynamics models are concerned with describing blood flow within the inner retinal or choroidal circulations and often rely on the Hagen-Poiseuille equation to determine flow in vascular segments.
This equation is a simplification of the more comprehensive Navier-Stokes equations made by making a number of assumption (see \textit{Additional information}).
The Hagen-Poiseuille equation states that blood flow ($Q$) in a vessel of length $L$ and radius $r$ is driven by a pressure drop ($\Delta p$) according to:
\begin{equation*}
  \label{eq:Hagen-Poiseuille}
  Q = \mathcal R\times\Delta p \mbox{, where } \mathcal{R} = \frac{8\mu L}{\pi r^4},
\end{equation*}
where $\mu$ is the apparent viscosity of blood which may account for changes in vascular resistance ($\mathcal R$) in vessels of varying diameter (see \textit{Additional information} and Figure~\ref{fig:effectiveViscosity}).
The fourth power of the radius in the formulation of vascular resistance describe the strong effect that even small contractions or dilations of a vessel can have on blood flow.
This relation between radius and flow is at the source of the autoregulation ability of blood vessels, which adapt their radius in response to a number of different cues.~\cite{Kur_2012}
The cellular constitution of retinal vessels suggests that blood flow is mainly regulated by arteries and arterioles.~\cite{An_2020,Kur_2012}
Evidence suggests that the choroid is also able, maybe to a lesser degree, to regulate blood.~\cite{Polska_2007,Riva_1997}
Neurological regulation (regulation by the autonomic nervous system) of choroidal blood flow has been suggested, in view of the rich innervation of choroidal vessels.~\cite{BeharCohen_2020,Polska_2007}

\textit{In silico} models provide valuable insights into the mechanisms that combine to provide healthy perfusion or, conversely, create conditions for the onset of retinal pathologies.
The assumptions made in these models can be validated by comparing simulation results and \textit{in vivo} measurements of changes of blood flow or vessel radii.
Vessel radius in the inner retina can be measured from retinal scans such as fundus photographs, fluorescein angiograms or OCT angiograms.
Blood flow can be measured using devices such as Doppler flowmeter or Doppler Fourier-domain OCT, in combination with radius measurements.~\cite{DoblhoffDier_2014,Wang_2009}
The blood's saturation in oxygen can also be measured \textit{in vivo}.~\cite{Geirsdottir_2013}
Reliable measurement of blood flow or vessel radii in the choroid is more difficult, though changes in haemodynamics can be observed \textit{in vivo}.~\cite{Riva_1997,Scherm_2019}

\begin{tcolorbox}[title=Additional information -- Fluid flow modelling]
  \begin{itemize}
  \item Viscous fluids such as blood are modelled by a set of differential equations relating acceleration, velocity, convection, density and pressure of a fluid called the Navier-Stokes equations.
    Solving those equations can be problematic in complex geometries, for example in a large network of vessels.
  \item Those equations simplify to the Hagen-Poiseuille equation (see main text) when assuming that the fluid is Newtonian and the flow incompressible and laminar.
    The equation Hagen-Poiseuille equation is relatively simple to solve, even on large vascular networks.
  \item An incompressible flow has a constant density regardless of the pressure and the fluid's properties are the same in all directions, namely, it is an isotropic fluid.
  \item A Newtonian fluid is a fluid which has a constant viscosity for given pressure and temperature, regardless of the amount of shear it is subject to.
  \item A Newtonian fluid flowing within a pipe is called laminar when fluid particles (e.g., red blood cells, blood plasma) move along smooth path with no mixing between each path.
    This kind of flow is opposed to turbulent flow, which appears when the velocity of the flow goes over a threshold determined by a combination of the fluid's viscosity and density and the pipe's dimensions.
  \item The propensity of a fluid to flow in either fashion is summarised by the dimensionless Reynolds number.
  \item Laminar flows show a parabolic velocity profile across the pipe, or vessel, cross section, with peak velocity at the centre and null velocity at the walls.
  \item As demonstrated by F\r ahraeus and Lindqvist, viscosity decreases with vessel width as red blood cells align into a single file, as illustrated in Figure~\ref{fig:effectiveViscosity}.~\cite{Faahraeus_1931}
  \item Empirical viscosity laws are used to account for the so-called F\r ahraeus-Lindqvist effect and the dramatic increase in vascular resistance seen in capillaries smaller than red blood cells, an example of which is shown in Figure~\ref{fig:effectiveViscosity}.~\cite{Haynes_1960,Pries_1990,Secomb_2013}
  \end{itemize}
\end{tcolorbox}


\subsubsection*{In health}

The current understanding of the perfusion of the retina is still lacking.
A significant effort has been made in modelling the complex circulatory systems that perfuse the retina in order to complete it.

\paragraph*{Haemodynamics in artificial retinal vasculature}

Takahashi et al. proposed an arterio-venous dichotomously branching and symmetric network where all vessels within a given generation are identical.~\cite{Takahashi_2009}
The diameter and length of the branching vessels follows the principle of least energy described by Murray's law.~\cite{Murray_1926}
This theoretical model was used to describe the distribution of flow and other haemodynamic parameters across the normal retinal vasculature.
While clinical studies often report changes in the diameter of large retinal vessels in various conditions, its relationship with total retinal blood flow, a clinically relevant parameter, is unclear.
The network by Takahashi et al. has been adapted to further investigate the physiology of the retinal vessels, in conjunction with clinical experiments.~\cite{Aschinger_2017,Pappelis_2020}
One of those studies showed that changes in total retinal blood flow is not linearly related with dilation of larger vessels but that, in fact, the diameter of the capillaries has the strongest impact on vascular resistance.
The patterns of vasodilation tested, however, assumed that the capillaries are able to vary their diameter when the validity of such assumption is still unclear.~\cite{Kur_2012}

The other study used clinical information to create patient specific networks based on the method of Takahashi et al.~\cite{Pappelis_2020}
The model included a autoregulation component as a function of retinal perfusion pressure (pressure entering the CRA, estimated as a difference between systemic blood pressure and intraocular pressure), and predicted the blood flow response to increases in intraocular pressure or drops in retinal perfusion pressure.
This allowed prediction of the autoregulation capacities of retinas in patients treated for hypertension or in eyes subject to ocular hypertension.
It showed that the retinas of both healthy and hypertensive patients was closer to the lower limit of autoregulation, suggesting that the retina may be more sensitive to decreases rather than increases in perfusion.~\cite{Pappelis_2020}

\paragraph*{Effects of extravascular pressure on haemodynamics} %NOTE: this could be one section with the next one (Autoregulation pathways)

Lumped parameters models have been used to investigate the relationships between intra- and extra-vascular pressures and haemodynamics in the retina and choroid.~\cite{Chiaravalli_2021,Fawzi_2019,Guidoboni_2014a,Nelson_2017,Petersen_2022,Prudhomme_2021,Sala_2020,Salerni_2019}
In those models, similar vessels are sorted into compartments, e.g., CRA, arterioles, capillaries, venules, CRV and choroid.
The physical processes within compartments and the interactions between them are summarized by few parameters.
In analogy with electrical circuits, haemodynamics in each compartment is characterised by a resistance, the combined effect of the vascular resistance of a compartment's vessels, and a capacitance, the total amount of blood that can be stored in the same vessels.
Autoregulation mechanisms impact the capacitance of a compartment, while pressures affect the resistance coefficient.

Lumped paramteres models have been used to investigate the links between gravity-induced shifts in intracranial fluids, ocular fluids and pressures and the loss of sight observed in astronauts.~\cite{Nelson_2017,Petersen_2022,Salerni_2019}
Nelson et al. proposed a baseline model simulating increases in ocular pressures secondary to such fluid shifts.~\cite{Nelson_2017}
The simplistic model predicted relatively well the response of intraocular pressure to changes in the gravitational environment or tilt of the body.~\cite{Nelson_2017,Petersen_2022}
A more comprehensive model including interacting brain, body and eye compartments showed that the retinal circulation was relatively spared from hypoperfusion compared to the choroidal circulation in microgravity conditions.~\cite{Salerni_2019}


\paragraph*{Autoregulation pathways}

Several lumped parameters models have been developed to better understand the mechanisms behind autoregulation in physiological conditions.~\cite{Arciero_2008,Arciero_2013,Guidoboni_2014a}
Arciero et al. used a symmetric network similar to the one by Takahashi et al. to understand the information pathways leading to blood flow autoregulation in the retina.~\cite{Arciero_2008,Arciero_2013}
Both models assumed that only arterioles possessed the ability to contract to regulate flow.
These models were used to investigate the response of individual or combined autoregulation pathways to changes in perfusion pressure or oxygen consumption rate of the cells.
The earliest model investigated the hypothesis that arterioles can contract in response to the oxygen saturation of blood in the downstream venules.~\cite{Arciero_2008}
The study showed that such autoregulatory pathway could account for the experimentally observed response of perfusion to changes in consumption rates.
This mechanism, along with other known ones, have been modelled together in a subsequent study.~\cite{Arciero_2013}
In this later study, the contribution of each pathway to retinal perfusion was assessed.
The results showed that the response of arterioles to carbon dioxide concentration in the surrounding tissue and oxygen saturation in the venules was essential to retinal autoregulation.~\cite{Arciero_2013}

Guidoboni et al. looked at the interplay between intraocular pressure, retinal haemodynamics, systemic blood pressure and autoregulation.~\cite{Guidoboni_2014a}
Instead of the mechanistic description of autoregulation proposed in the work by Arciero et al., they used an empirical response function of arteriole diameter to changes in perfusion pressure.
Another difference with the previous work is the assumed capacity of the venules and central retinal vessels to dilate or collapse, passively and to some limited extent, in response to pressure gradients across their walls.
The retinal haemodynamics of six types of physiology (high, normal or low blood pressure, with or without working retinal autoregulation) were simulated across a range of intraocular pressure.
The model predicts the ranges of intraocular pressure within which each physiology can maintain relatively constant blood flow.
The dependence of these ranges to systemic blood pressure and autoregulation capacity may explain some discrepancies regarding the effects of intraocular pressure on retinal haemodynamics found in experimental studies.~\cite{Guidoboni_2014a} 

\paragraph*{Oxygen transport and spatial models}

The transport and delivery of oxygen to the tissue has been modelled by several groups.~\cite{Aquah_2021,Causin_2015,Liu_2009}
The earliest one used a short arteriolar tree, taken from a retinal photograph of a healthy young volunteer, and solved the Navier-Stokes equation to obtain haemodynamics of blood in these vessels.~\cite{Liu_2009}
The downstream vasculature was represented as a structured tree extending the outlets of the segmented arterioles.
This model reproduced the distribution of intravascular oxygen observed \textit{in vivo} and could serve as a baseline to analyse oxygen distribution in realistic vascular networks.
However, reproduction of the retinal vasculature is limited to larger vessels.
Therefore, the use of artificial vasculature with similar characteristic as the real one may be useful to model the whole extent of the retinal vasculature.

Causin et al. used the fractal similarity between retinal vessels and diffusion-limited aggregation process to generate a three dimensional network of three of the vascular plexi of the retina.~\cite{Causin_2015}
Oxygen transport throughout the extensive vascular network was modelled as well as the oxygen delivery to the tissue, modelled as a rectangular slab.
The good agreement with data validated the use of the artificial networks.
The effects of haemodynamics parameters on different parts of the tissue and the vasculature was analysed with the model and showed, among other things, the sensitivity of perfusion of retinal ganglion cells to blood viscosity and metabolic consumption.~\cite{Causin_2015}


Several other studies have used vessels reproduced from images to analyse the distribution of blood flow and the influence of the morphology.~\cite{Malek_2014,Malek_2015,Rebhan_2019}
Malek et al. tried to characterise the flow distribution in reproduced arterioles and veins to quantify the impact of vessel tortuosity on haemodynamics parameters.~\cite{Malek_2014,Malek_2015}
The deformation of vessels caused by the pulse of blood is a computationally complex task to model and a simplified model has been developed and applied to a similar reproduction of the retinal vasculature, with a brief analysis of its effects on haemodynamics.~\cite{Aletti_2016}
Notably, Rebhan et al. have looked at the interactions of retinal haemodynamics and tissue stress using reproductions of the large vessels of healthy and diseased eyes, an aspect previously overlooked that may have importance in, e.g., glaucoma and diabetes.~\cite{Rebhan_2019}

The use of realistic vascular networks provides a link between haemodynamics and clinically relevant indices.
The effect of tortuosity of the vasculature has been investigated on short, segmented sections of the retinal veins.~\cite{Malek_2014}
In this model, the Navier-Stokes equations are used to find flow within the arterioles, while peripheral circulation is accounted for by an impedance condition at the outlets of the segmented tree.

%%% COULD BE DELETED?
\textit{In silico} models have also been developed to understand how the retinal vasculature develops.
A review of some of those models can be found in other reviews.~\cite{Arciero_2019,Roberts_2016}
However, the role haemodynamics plays in the angiogenesis of the vascular beds is still elusive.
Bernabeu et al. proposed a model of blood flow in a reconstructed murine vasculature that can be used to simulate the haemodynamics forces that cannot be measured in small capillaries but may affect angiogenesis.~\cite{Bernabeu_2014}
Others looked at how the branching patterns in the early development of the murine retina may affect later development.~\cite{Mirzapour_Shafiyi_2021}
The results showed that hyper-branching behaviours in the young vascular bed reduces the blood supply, and hence the oxygen, at the growing ends of the vasculature.
The lack of oxygen on this front may drive up-regulation of vascular growth factors promoting vascular growth.


\paragraph*{Haemodynamics in the central retinal vessels}

Others have used finite element modelling, a technique suitable for modelling of mechanistic forces, to understand the blood flow in the central retinal vessels.~\cite{Guidoboni_2014,Jin_2020}
The central retinal vessels enter the retina along the optic nerve where they are under pressure from the surrounding tissue, the cerebrospinal fluid surrounding the eye, the intraocular pressure and, potentially, the artery and the vein may also compress each other.~\cite{Nickells_2012}
The lamina cribrosa's purpose is to act as a buffer protecting the vessel where all those pressures confront each other.
Guidoboni et al. simulated the displacement of the lamina cribrosa and the stress due to different intraocular pressures and pressure induced by the cerebrospinal fluid.~\cite{Guidoboni_2014}
The predictions of blood flow in the CRA showed good agreement with measurements made during intraocular pressure elevation, suggesting that the observed decrease in blood flow velocity is caused by the lamina cribrosa compressing the artery.
Note that this model assumed a constant blood pressure, average of the systolic and diastolic blood pressures.
Pulsatility may have a significant impact on intraocular pressure and, consequently, blood flow in the central retinal vessels and has been investigated within a more comprehensive model of the ocular structures.~\cite{Jin_2020}


Whether the vascular plexi in the retina, namely, the superficial, intermediate and deep plexi, are connected \textit{in series} or \textit{in parallel} is still unclear.
The \textit{in parallel} configuration suppose that both arterial and venous connections exist between plexi.
Conversely, in the \textit{in series} configuration, blood inflow comes solely from the superficial plexus, while venous drainage happens only in the deep plexus.
We refer the reader to Figure 1 in the paper by Chiaravalli et al. for a schematic of those two configurations.\cite{Chiaravalli_2021}
The lumped parameter models developped in this paper simulate haemodynamics between the CRA to the CRV in five vascular compartments.~\cite{Chiaravalli_2021}
The authors modelled both configurations and compared the responses of each plexi to intraocular pressure elevation and occlusion of the CRV.
The model showed that the \textit{in series} configuration better captured the response observed in clinical studies, and may better describe the actual physiology of the retinal vasculature.

\paragraph*{Choroidal circulation}
While the inner retinal vasculature feeds the inner third of the retina, the remaining two thirds are perfused by the choroidal circulation.
In contact with the retina is the CC, innermost layer of the choroid.
The CC provides oxygen across the Bruch's membrane to the photoreceptors and other neuronal cells of the outer retina through diffusion only.
Therefore, a high vascularisation is necessary to provide enough oxygen despite the distance.
Defects in choroidal blood flow are associated with major retinopathies such as neovascular AMD and DR~\cite{Pemp_2008}.
Despite its importance, little is known about the physiology of the choroid.
Likewise, little work has been done on modelling choroidal circulation.
Early work tried to decipher whether the choroid of rabbits was able to regulate its flow in response to changes in systemic pressure.\cite{Kiel_1992}
Simulations in pair with experimental evidence suggests a autoregulatory reflex in the choroid triggered by blood pressure.
Since then, only Zouache et al. modelled the physiology of choroid and its peculiar architecture~\cite{Zouache_2015}.
By simulating the lobular structure of the CC, they investigated the effects of the geometry of lobules on the flow of blood.
Their work suggested that the distribution of flow separators in the CC and the location of inlet and outlets in individual lobules may explain the localisation of oedema or neovasculature in diseased eyes.\cite{Zouache_2015}



\subsubsection*{In disease}

%As mentionned previously, many retinal pathologies have or may have a link with irregular haemodynamics.
In disease, the haemodynamics of the retinal circulations may change drastically, however, the exact aetiology of vascular retinal diseases is still elusive.

\paragraph*{Models of pathological geometries}

The model by Rebhan et al. previously discussed compared haemodynamics in vascular networks segmented from a healthy eye and eyes affected by glaucoma or diabetes.~\cite{Rebhan_2019}
The embedding of the vessel in the tissue revealed an increase in wall shear stress in the diseased eyes compared to the healthy one.
However, it was noted by the author that the absence of the downstream vasculature is likely to affect the quality of predictions.\cite{Rebhan_2019}
Along the same lines, higher vascular tortuosity, a common sign of ageing and disease, has been shown computationally to increase the pressure drop across the vasculature.~\cite{Malek_2014}
However, vessel-tissue interactions were not considered in the model.

Eyes with visual impairments such as myopia progressively affect the shape of the retina and may develop neovasculature.~\cite{Medina_2016}
This suggests a link between the curvature of the eye and retinal perfusion.
However, the effects of curvature have been assumed insignificant in most haemodynamics models to date.
Dziubek et al. modelled this curvature by representing the retina as a thin, curved surface, within which an artificial network of vessels is embedded.~\cite{Dziubek_2015}
Interestingly, vessels are considered as pores in the tissue and, therefore, blood flow was modelled using Darcy's law, which describes flow through a porous medium.
Therefore, arteriolar, capillary and venular networks are kept virtually separated but communicate through a `hierarchical' velocity variable, as opposed to the spatial velocity which transports blood throughout the retinal surface.
The dichotomous tree proposed by Takahashi et al. was used to generate the networks.
The model confirms clinical suspicions of a change in retinal haemodynamics due to ocular curvature.
Furthermore it highlighted the non-uniform effects on the retina, with the temporal region being less affected by ocular shape.\cite{Dziubek_2015}

\paragraph*{Models of glaucoma and vessel occlusion}

Numerous models have been developed to explain alterations of blood flow observed in eyes subject to glaucoma or vessel occlusion.\cite{Chuangsuwanich_2016,Guidoboni_2014,Sala_2018,Sala_2020}
Elevated intraocular pressure, a hallmark of glaucoma, is expected to affect the retinal circulation, causing loss of sight.
Modelling of the interactions between the ocular structure surrounding the central retinal vessels has shown the role of a stiffened lamina cribrosa resulting from such conditions.~\cite{Guidoboni_2014}
In addition, the model showed that the geometry of the sclera and the lamina cribrosa affect the sensitivity of blood flow to elevation of intraocular pressure.
This framework was extended in Sala's thesis to create the Ocular Mathematical Virtual Simulator, a simulation environment for the interactions of haemodynamics and biomechanics in the eye.~\cite{Sala_2018,Sala_2020}
The environment allowed to simulate haemodynamics of individual patients characterised by a handful of parameters, yielding significantly different results for each combination of parameters.
Results showed that high intraocular pressure can cause collapse of the CRV and major differences in the displacement of the lamina cribrosa.
Interestingly, it showed that the perfusion of the lamina cribrosa is also negatively affected by ocular hypertension, a potentially significant mechanism in the understanding of glaucoma.~\cite{Sala_2020}
Perfusion and haemodynamics within the lamina cribrosa has been modelled separately using a large number of artificial capillary networks statistically representative of different morphologies of the lamina cribrosa.~\cite{Chuangsuwanich_2016}
For a review of the use mathematical models in glaucoma research, we refer the reader to the review by Harris et al.~\cite{Harris_2013}

\paragraph*{Models of diabetic retinopathy}

Microaneurysms are an early manifestation of DR where the walls of capillaries form outpouchings disrupting blood flow and which are susceptible to rupture.
We found four studies investigating haemodynamics in reconstructed microaneurysms using computational fluid dynamics models.~\cite{Bernabeu_2018,Czaja_2022,Li_2020,Li_2022}
One looked at the shape of microaneurysms and how it influences haemodynamics, and in particular shear rates, in an attempt to determine predictors of the likelihood of leaking blood clotting.~\cite{Bernabeu_2018}
Similarly, Czaja et al. investigated wall shear stress in microaneurysms in the event of stiffened red blood cells, another symptom of diabetes.~\cite{Czaja_2022}
A notable difference with the previous model, however, is the use of cell resolved blood flow simulations, where individual cell are modelled and transported by blood flow.
This allowed to investigate separately the flow of red blood cells and platelets cells and showed the differences in penetration through the microaneurysm between the two cell types.
The stiffened red blood cells were found to induce higher wall shear stress, both in the aneurysm sac and in the vessels feeding and draining it.
In two studies, Li et al. have also simulated the flow of red blood cells and platelets in microaneurysms with a focus on the platelet flow through the microaneurysm, as they may be linked with the formation of blood clots.~\cite{Li_2020,Li_2022}


Panretinal photocoagulation therapy is a common procedure to treat the ischaemia in retinae with proliferative DR.
While practice has shown the efficiency of the procedure, its mechanisms of action remain unclear.
It is thought that procedure treats retinal ischaemia by reducing the oxygen consumption of the photoreceptors affected by the laser-induced burn.~\cite{Fawzi_2019,Gast_2016}
However, effects of the therapy on blood flow has been observed.
With a simple lumped parameter model and the assumption that panretinal photocoagulation increases vascular resistance in the periphery of the macula, Fawzi et al. showed how the surgery could increase macular blood flow.~\cite{Fawzi_2019}
The increased perfusion in the macula would then explain the decrease in VEGF concentrations and the subsequent regression of neovasculature.
The models predictions are along the lines of the seemingly increased flow in macular capillaries.~\cite{Fawzi_2019} 
The patterns of burn used in panretinal photocoagulation is also subject to debate.
Gast et al. investigated the effects of different patterns on the propagation of ischaemia due to capillary occlusion.~\cite{Gast_2016}
The model showed how different patterns of burn affect the propagation of ischaemia and suggest that targeting the non-ischaemic peripheral retina with appropriate patterns may be effective at containing it.


\subsection*{Anti-VEGF}

In proliferative DR and neovascular AMD, loss of sight is linked to detachment of the retina and accumulation of fluids or oedema~\cite{Roberts_2020, Waldstein_2016}.
Those are consequences of the growth of leaky blood vessels, referred to as neovasculature, in the neuronal retina and, in particular, in the macula.
The pathological angiogenesis is driven by gradients of VEGF which are upregulated by hypoxia (a lack of oxygen) in the diseased retina.
Binding of free VEGF molecules to their receptors on the walls of existing blood vessels triggers the migration of the endothelial cells constituting said vessel walls.

The treatment of neovasculature is predominantly done via frequent injections of VEGF-inhibiting molecules that bind to the free VEGF present in the retina, ultimately inhibiting the angiogenic process.
These injections are typically done directly in the vitreous humor of the eye with a needle, though alternative delivery techniques are being investigated~\cite{Kim_2021}.

Molecules present in the vitreous are naturally eliminated through the aqueous humour flow, referred to as the anterior clearance route.
The posterior clearance route refers to clearance through the choroidal circulation.
In addition, the ILM and RPE, respectively on the inner and outer retina, act as barriers to the molecules~\cite{Park_2015}.
Therefore, the presence of the drug in the retina and the choroid is limited to a fraction of the injected dose. 

Despite its general efficacy, current treatment strategies are sub-optimal for some patients in terms of dosage and interval between injections.
Eyes showing little to no improvement to their condition are referred to as non-responsive and present a real challenge for clinicians.

Furthermore, VEGF induced angiogenesis is a natural response to inflammation and hypoxia, in the eye and the rest of the body. 
Therefore, repeated injections pose a number of problems.
Firstly, the intravitreal injections (IVI) can cause further inflammations within the retina, triggering additional VEGF upregulation~\cite{Iyer_2022}.
Secondly, with the current doses, the unbound anti-VEGF molecules that are cleared from the eye are found in significant levels in the systemic circulation, raising concerns about the safety of IVI.
Indeed, while it is still matter of debate, it has been suggested that IVI of anti-angiogenic molecules could be linked with serious adverse effects including haemorrhages and strokes~\cite{Avery_2016, Kaiser_2019, Maloney_2021}.

Knowledge of the determinant of the total exposure of the retina and the CC to the drug is important to develop better therapeutics molecules and administration strategies and reduce risk and burden on the patient.

However, while aqueous and vitreous humors can be sampled \textit{in vivo}, the concentrations in the retina and choroid remain unknown. 
Therefore, estimates of the retinal kinetics of molecules are often based on either animal experiments or on samples of the aqueous humor, the vitreous humor and systemic plasma.
Mechanistic PK and PD models can help overcome the issue of the lack of \textit{in vivo} data in the retina and provide insight into the complex true relationships between drug characteristic and physiological parameters.
This section is dedicated to computational models of VEGF and its inhibitor, whether individually (PK models, VEGF production models) or interacting together (PD models).

\paragraph*{Compartmental PK/PD models}

Often in PK analysis of ocular or systemic fluid, the half-life of a molecule is estimated by fitting exponentially decaying curves to the data in order to compute the total exposure to the drug (area under the concentration-time curve) and maximal concentration~\cite{Bakri_2007, Kaiser_2019, Park_2015, Park_2016, Xu_2013}.
This assumes that the clearance rate of molecules in the eye is proportional to the concentration of said molecule at all times and ignores potential effects of interactions with the tissue or the presence of multiple clearing pathways (e.g., the aqueous humor outflow and the choroid circulation).
Understanding the determinants of the PK of the large anti-VEGF molecules in the eye is essential to develop more efficient molecules and such simple models do not provide this kind of insight.

Mechanical models can be used to determine the relationship between physiological and drug parameters and estimate their \textit{in vivo} value, which may differ from the theoretical or \textit{in vitro} values.
Analytical relationships between a molecule's characteristics and its ocular half-life can be derived from those models and can help design of longer lasting therapeutics.

In a series of papers, Hutton-Smith et al. used two (vitreous and aqueous humors) and three (including the retina) compartments models to estimate the true, \textit{in vivo} parameters of IVI of anti-VEGF~\cite{HuttonSmith_2016,HuttonSmith_2017,HuttonSmith_2018}.
Using data on rabbit eyes, they demonstrated that the half-life of those molecules in the eye is proportional to the cubic root of their hydrodynamic radius, namely,
\begin{equation}
  \label{eq:t12_Hutton-Smith}
  t_{1/2} = \alpha\sqrt[3]{R_h},
\end{equation}
which agrees with reported experimental values~\cite{HuttonSmith_2016}.
This work also highlighted the difference between \textit{in vitro} and \textit{in vivo} binding rates of anti-VEGF to VEGF, which could be of multiple order of magnitudes compared to values used in previous similar PD models~\cite{Saunders_2015}.

Using the two-compartment PK model by Hutton-Smith et al., with values of the hydrodynamic radius, ocular half-life and vitreous radius collected for this meta-analysis, Caruso et al. found a slope $\alpha=2.1$ in Equation~\ref{eq:t12_Hutton-Smith}~\cite{Caruso_2020}.
This result differs from the theroretical value derived by Hutton-Smith et al., reported at $\alpha=4.4$, computed based on $t_{1/2}$ computed in previous PK analyses.~\cite{HuttonSmith_2016}
The discrepancy may be explained by the consideration of choroidal clearance in the more recent work, a mechanism which was ignored in the earlier models, ergo suggesting that posterior clearance should be included in PK models of IVI.


Bussing et al. proposed to extend a compartmental model of the whole rabbit body, connected through blood flow and lymphatic circulation, with an eye compartment subject to IVI of anti-VEGF~\cite{Bussing_2020}.
The comprehensive model comprises over a hundred compartments, with all exchange rates and reaction rates determined using values reported in the literature.
The results showed quantitative and qualitative agreement with experiments without necessitating determination of unknown parameters. 

The previous PKPD models assumed a constant with time and homogeneous production rate of VEGF.
However, an \textit{in silico} model of an \textit{in vitro} setup on the RPE suggests that spatial arrangement of RPE cells and atrophied tissue play an important role on the production of VEGF that may explain the progression of AMD into its neovascular form~\cite{Baker_2017}.  

\paragraph*{Finite element PK models}

While insightful on the relationship between ocular availability and molecular characteristics, compartmental models assume that VEGF, anti-VEGF and their bindings are well-mixed within each compartment.
However, ocular fluid flow can impact the delivery of drug to the retina.
Flows may be influenced by the structure of the eye and may differ strongly between species.
A number of groups have developed finite element models of the eye, adding the contributions of ocular fluids, structure, heat and gravity to the PK analysis~\cite{Lamminsalo_2018, Missel_2012, Zhang_2018}.
Such models can help make better use of experiments on animal eyes by providing a framework to translate data from one species to another.
Some of these models are reviewed here and a comprehensive review of those models and associated findings can be found in the review by Missel and Sarangapani~\cite{Missel_2019}.

Zhang et al. used a simplified three-dimensional representation of the vitreous and the retina to investigate the distribution of molecules injected in the vitreous or under the choroid~\cite{Zhang_2018}.
Their results support the well-mixed hypothesis of intravitreally injected molecules by showing the small effect the initial mixing has on the concentration-time profile.
Furthermore, the model showed that suprachoroidal injections were not suited for delivery of large molecules.

Missel et al. created physiologically accurate three-dimensional geometries of the rabbit, monkey and human eyes to investigate the effect of inter-species structural differences on drug clearance~\cite{Missel_2012}.  
They demonstrated the importance of the canal of Petit in the clearance of substances from the aqueous humor, particularly for molecules with slow diffusivity. 
Furthermore, they showed that an increase in intraocular pressure, within a normal range of \SIrange[range-units = single]{10}{20}{\mmHg}, significantly decreases the passage rate of the larger molecules from the vitreous to the aqueous.
By accurately modelling the eye of different species, including humans, they aim to offer a framework to translate experimental data from one species to another~\cite{Missel_2012}.

Lamminsalo et al. used the geometry of the rabbit eye designed by Missel et al. to investigate the contributions of anterior (through the aqueous humor outflow) and posterior (through the choroidal circulation) routes in the clearance of intravitreally injected macromolecules similar to anti-VEGF molecules~\cite{Lamminsalo_2018}.
Their model suggested that only \SIrange[range-units = single]{5}{24}{\percent} of the injected drugs is eliminated through the retina, in accordance with previous modelling work~\cite{HuttonSmith_2017}.
However, this percentage increases with intraocular pressures, which might elevate with age and other systemic variables~\cite{Armaly_1967,Hashemi_2005}.
Furthermore, diffusion of the injected molecules through the retina, the RPE and the Bruch's membrane is not yet clear and may influence the posterior clearance rates.
In later work, the group extended the previous model to estimate the diffusion coefficients in those tissues using \textit{in vivo} PK data and found a RPE permeability similar to the one found by Hutton-Smith et al.~\cite{Lamminsalo_2020,HuttonSmith_2017}.

\paragraph*{Models of treatment outcome}

Other models have simulated the effect of anti-VEGF therapy on clinically relevant features of neovascular AMD, e.g., visual acuity and size of macular oedema~\cite{Edwards_2020, Hoyle_2017, Mulyukov_2018}.
These models can be compared directly with clinical trials and, potentially, can be used to run \textit{in silico} trials, as discussed in Section~8. %~\ref{sec:InSilicoTrials}.

Using a compartmental modelling approach, Hoyle and Aslam showed that their model could reproduce the results from landmark studies of anti-VEGF therapy in neovascular AMD~\cite{Hoyle_2017}.

Mulyukov et al. attempted to model the response to therapy in terms of visual acuity with a mixed effect model~\cite{Mulyukov_2018}.
The model was calibrated on a large dataset compiled from various clinical trials and captures the average trends of treatment response without using any patient specific information other than visual scores and treatment strategy~\cite{Mulyukov_2018}.

Building on this, Edwards et al. added the buildup of tolerance to the drugs and its effect on visual outcome, using spatial and non-spatial models~\cite{Edwards_2020}.
After fitting the model-specific parameters, both models show good agreement with the observations from the clinical studies.
However, the spatial model showed better performance at predicting treatment outcome of a patient non-responsive to treatment.





\section*{Drug delivery to the retina}

In this section we review models aimed at optimising or developing drug delivery techniques for the treatment of retinal diseases.
Ocular drug delivery, including delivery to the retina, has recently been reviewed elsewhere from a computational fluid dynamics point of view.~\cite{Bhandari_2021}
Thus, this section will focus on models that were not covered in said review.

\subsection*{Intravitreal injections}

\textit{In silico} models of the PK and PD of anti-VEGF molecules, mainly administered inside the vitreous, have been discussed in Section 5.
Here, we focus more specifically on models aimed at understanding and optimising the surgical procedure of injecting drugs in the vitreous humor with a needle and how the outcome may be affected by the eye's characteristics.

The effects of injection parameters such as needle shape, angle of insertion, speed of injection may affect the delivery of drug to the retina and are hard to assess \textit{in vivo}.
Several models have been developed to identify those effects using finite element realistic geometries of the eye.
Some studies assumed the initial dose to be a sphere with drug concentration equal to the dose.\cite{Friedrich_1997,Friedrich_1997a}
In those studies, the initial location of the dose showed to affect the clearance of the drug from the vitreous by almost four-fold for the configurations tested.\cite{Friedrich_1997}
However, flow in the vitreous was neglected.
Therefore, drug transport was solely due to diffusion and thus strongly dependent on the dose.\cite{Friedrich_1997}

Needle type, injection speed and penetration angle of the needle were accounted for to describe the initial concentration profile in another finite element model of the human eye.\cite{Jooybar_2014}
In this model, the drug was transported by flows in the vitreous as well as diffusion.
Furthermore, injection parameters affected the shape of the initial distribution of drug, although the needle was not explicitly modelled.
The model showed significant sensitivity to the injection parameters on the exposure of the macula to the drug.
Slower injections and larger needle gauge were shown to increase by an order of magnitude the concentration peak at the macula compared to a model assuming a spherical initial distribution of the dose.
Unsurprisingly, angle of penetration affects strongly the concentration peak, by up to \SI{80}{\percent} at the macula.\cite{Jooybar_2014}
This model showed the importance of considering advective transport of intravitreally injected drugs in the vitreous. 

Injection parameters may also augment the risks of complications due to the procedure.
By modelling the eye, the needle and their mechanical interactions, including deformation of the cornea, it was found that an angle of \SI{45}{\degree} between the needle and the optical axis was ideal to minimize complications.\cite{Karimi_2018}
To draw this conclusions, the authors assumed that post injection complications were correlated with the maximal principal stress, namely the normal stress on a plane subject to no shear stress.  
The increased stress caused by insertion angles closer to the vertical or horizontal axis seem to correlate with experiments which reported more injuries for those angles.\cite{Karimi_2018}

With age, the vitreous humour undergoes changes causing its liquefaction.
Furthermore, in disease, it may be replaced by a substitute gel or oil in order to lower pathological traction at the interface with the retina.
Alterations of the properties of the vitreous humour, or its substitutes, has been investigated computationally with similar finite element models, where drug transport is modelled by a advection-diffusion equation.\cite{Kathawate_2008,Modareszadeh_2012}
Point sources for the injection of the dose are used in those studies.

The earliest model investigated the possibility of toxic levels of drugs in the retina.\cite{Kathawate_2008}
The model showed that the exposure of the retina increases strongly in configurations with low diffusivity of the drug and low viscosity of the vitreous substitute, due to convection overtaking diffusion.\cite{Kathawate_2008}
Shifts to predominantly convective transport of drug has been showed to happen due to saccadic movements typical of vitrectomised eyes.\cite{Modareszadeh_2012}
The finite element model showed that higher movement amplitude hasten the spread of the drug within substitutes of vitreous humour.
While homogenisation of drug concentration was reported to happen in a time scale of days, this may reduce to the order of minutes in vitrectomised eyes.\cite{Modareszadeh_2012}
Furthermore, it was showed that the diffusion coefficient of the drug had limited impact on its spread in the vitreous after these initial few minutes.\cite{Modareszadeh_2012}

More recently, saccadic movements effects on intravitreally injected drugs have been simulated in a vitreous liquefied with age.\cite{Ferroni_2020}
This model predicted that, in the presence of saccades, the drug concentration homogenised throughout the vitreous in less than a minute when it takes about a day otherwise.
Interestingly, it has been shown computationally that in locally liquefied vitreous (e.g., a substitute for the vitreous inserted surgically), fluid flow converges towards the liquefied region, as it offers less resistance.\cite{Khoobyar_2022}
This result is of interest in understanding the kinetics of intravitreally injected drugs, especially larger ones such as anti-VEGF which are more subject to convection.

\subsection*{Implants/port-delivery}

A number of models, including those previously reviewed in this paper, have been used to compare the efficacy of drug delivery to the vitreous by an injection or by a controlled release from a system implanted in the eye, typically in the vitreous or on the outer surface of the sclera.\cite{Jooybar_2014,Kathawate_2008,Kavousanakis_2014,Park_2005}
Overall, these comparisons highlighted the capacity of controlled release systems to prolong the drug availability in ocular tissue.
However, those models were not concerned with the mechanisms of drug delivery from those implants but rather assumed empirical release rates.

Understanding of the degradation process of vitreal implants is essential to control drug release.
The effects of altered vitreous on this process has been modelled.\cite{Ferreira_2020}
Degradation process of the implant and drug transport in the vitreous and retina were coupled in this model.
Drug distribution profiles were simulated for two different vitreous humour substitutes, namely a silicone oil and a saline solution.
The authors concluded that silicone oil substitutes could delay the degradation of the implant and provide higher mean concentration in the retina.
On the other hand, the saline solution substitute showed similar or lower concentrations compared to non-vitrectomised eyes.\cite{Ferreira_2020}

In the model by Li et al., the drug molecules are trapped within the inner mesh structure of a hydrogel, represented as a sphere.\cite{Li_2022a}
Unlike the previous chemical degradation of the implant, degeneration of the hydrogel corresponds to loosening of the mesh over time, described as an empirical exponential law.
Both the initial location and properties of the hydrogel were varied.
Perhaps unsurprisingly, while the location did not affect the depletion of drugs from the hydrogel, a position closer to the target site, e.g., the macula, caused higher and earlier peaks in concentration.
However, higher peaks implied quicker clearance from the macula and therefore concentrations reach similar levels as other implantation sites within two weeks.\cite{Li_2022a}
Hence, the location of the hydrogel has to be chosen wisely as to not induce toxicity while maintaining therapeutic levels of drug within the retina.
In contrast, the hydrogel properties tested did not show a significant effect on macular concentrations, though initially tighter meshes cause a delay in the release of the drug.

Recently, a PK model specific to anti-VEGF molecules in the vitreous and aqueous humour, released from degrading spheres, has been simulated and validated against experimental data.\cite{Heljak_2022}
Note that, unlike some of the PK model of anti-VEGF described in the previous section, those models did not ignore convective transport in the vitreous or the other ocular tissues.
Rather, those layers are considered porous mediums through which transport is driven by diffusion and a pressure gradient between the intraocular pressure in the anterior part of the eye and the lower pressure at the sclera.~\cite{Ferreira_2018,Ferreira_2020,Heljak_2022,Khoobyar_2021,Li_2022a}
Noteworthy is the work of Khoobyar et al. to determine the depletion of drug from the an implant.~\cite{Khoobyar_2022}
Through rigorous mathematical analysis of a simplified drug transport model, they derived an analytical formulation for the estimated half-life of an implant in the vitreous which depends on the ratio of convective mass transfer to diffusivity, namely the mass-transfer Biot number.
However, this estimate is valid only when diffusive transport dominates over convective transport.



The same equations can be used to describe the transport of molecules from the sclera to the vitreous, a scenario corresponding to implants inserted on the outer surface of the sclera.
The case of a drug diffusing through such implant to enter the sclera, later reaching the choroid and retina, was modelled \textit{in silico} recently.~\cite{Abootorabi_2021}
The clearance through choroidal and retinal circulation was accounted for and simulations showed very good agreement with data.
The model revealed the influence of the implants porosity on the controlled release of drugs and suggests parameters could be tailored to individual needs.~\cite{Abootorabi_2021}
Kotha and Murtom\"aki also modelled drug release from a sclera implant, with the difference that choroidal blood flow was modelled explicitly.~\cite{Kotha_2014}
The influence of diffusion coefficients in the sclera as well as the permeation coefficients regulating exchanges between each layer of tissue were quantified and demonstrated complex relationships between the parameters and the efficacy of the implant.

Of particular relevance to scleral implants is the retinal barriers to drug.
The role of these barriers, namely the RPE-Bruch's membrane complex, choroidal and retinal circulation and the ILM, has been investigated \textit{in silico}.
Active transport by the RPE, along with clearance from the inner retinal blood vessels, were included in a model by Causin et al.~\cite{Causin_2016}
The permeability of the blood vessels walls was shown to have little influence on clearance, though it is suggested by the authors that this could be a consequence of the simplifying assumptions made concerning mass transport between the retina and its vasculature.
In contrast, active transport by the RPE was found to have a significant impact on drug concentration in the retina and vitreous.
Other models came to the same conclusion, despite this specific transport mechanism being modelled differently.~\cite{Balachandran_2008,Kotha_2014}
Despite this evidence, active transport across the RPE has generally be neglected in PK models of IVI and controlled-release devices.


\subsection*{Subretinal, periocular and systemic administration}

While accumulating evidence suggests otherwise, the retina and the eye in general are thought to be isolated from systemically injected drugs, on account of the numerous biological barriers separating the two.
Therefore, systemic, or intravenous, administration of drugs for treatment of retinal pathologies remain uncommon.
Accordingly, few modelling works have been published on the matter.
However, understanding of the PK of intravenously injected drugs remains of interest since harmful effects on the retina have been reported.\cite{Fu_2017}

In fact, significant ocular exposure, determined by a non-compartmental model, to intravenously injected antibodies has been reported.\cite{Shivva_2021}
The possibility to deduce vitreous concentrations after intravenous injection has been demonstrated by a compartmental model.\cite{Vellonen_2015}
However, the transfer rate from systemic circulation to the eye compartment was taken from a previous model and therefore represents the rate of clearance from the eye compartment into the systemic circulation, which may differ from the clearance rate in the opposite direction.
The same model applied to the analysis of experimental data on the permeability of the outer blood-retinal barrier found asymmetric exchange rates between the choroid and the vitreous.\cite{Ramsay_2019}
Additionally, the analysis suggests that the RPE may not be the main route of clearance from the vitreous for drugs entering the retina via the systemic circulation.\cite{Ramsay_2019}

In contrast, injections into the subretinal space, have gained traction, in particular for the delivery of cell or gene therapy.
Subretinal injections provide direct access to the targeted cells, namely the photoreceptor and RPE cells.
In both cases, the blood-retinal barrier plays an important role in the total exposure of the tissue to the therapeutics.
Yet, the mechanisms of exchange between systemic circulation and the retina remain elusive and few \textit{in silico} models of subretinal or systemic administration of ocular drugs have been developed.
A computational fluid dynamics model for the transport of molecules across the RPE was developed and calibrated with an \textit{in vitro} experimental setup.~\cite{Davies_2020}
Despite active transport not being considered in the model, this work provides a validated framework to build upon to incorporate such dynamics and to determine key parameters of the RPE.

Injection of anti-VEGF under the sclera (periocular), as a safer and less invasive alternative to the intravitreal route, is another promising technique.
The method would also benefit from understanding of the ocular barriers.
A single dimensional, time-dependent, diffusion model was developed to simulate the transport of a protein across those layers in the mouse eye after periocular injection.~\cite{Gabhann_2007}
The diffusivity of each layer was estimated based on their fraction of extracellular space, while permeability of the barriers and clearance rates were fit to experimental data.
The model showed it only takes 75 minutes for most of the injected dose to enter the eye and 95 minutes for it to be cleared, with over \SI{99}{\percent} of the clearance happening through the choroidal and episcleral circulations.
Compared to simulated IVI, periocular injections showed a two-fold higher peak concentration in the retina.
However, the peak persists longer with IVI.
Somewhat similar work but modelling injections of anti-VEGF molecules in the space between the choroid and the sclera showed that clearance rate from the episclera was negligible for large molecules.~\cite{Zhang_2018}
However, it may still play a role for smaller ones such as the fluorescent protein simulated in the previous model.~\cite{Gabhann_2007}

Of potential interest to the reader are two models of topical delivery and spray systems.~\cite{Mori_2017,Nweze_2020}
Modelling work of these techniques remains scarce but the growing interest for, e.g., cell and gene therapy may motivate modellers in the coming years.



\section*{\textit{In Silico} Clinical Trials in the Retina}

The previous sections have detailed the vast literature that deals with modelling retinal haemodynamics, oxygenation, and the impact of drugs in both healthy and diseased retinas. However, there is currently very little research on formalising these models within the framework of in silico clinical trials (ISCTs). ISCTs are effectively simulations of clinical trials to test medical devices or drugs with the aim of eventually being used as digital evidence to reduce, refine and replace animal and human participants in preclinical and clinical experiments~\cite{Viceconti2021a}. At each stage of clinical trials (preclinical, Phase I, II, III), ISCTs can be used: for early proofs-of-concept in the drug development phase; to simulate greater numbers of patients in each phase hence representing greater biological variability in the trial; to augment trials such that fewer real-world patients are required; to run numerous trials that can optimise the intervention that wouldn’t be possible in reality; as well as to investigate rare events that might not occur in a real-world trial~\cite{Pappalardo2019, Viceconti_2016, Viceconti2017}. Another crucial aspect of an ISCT is the ability for a patient to act as their own control simulation-wise, introducing a new potentially more impactful/patient-specific way of running clinical trials.

The paradigm of an ISCT is very similar to that of a real-world clinical trial except the disease, intervention and trial setup is simulated. ISCTs require virtual populations of the disease of interest; a mechanistic model of the disease and treatment using the proposed intervention; and a statistical analysis plan that will analyse the output of the trial~\cite{Alfonso2020}. Figure~\ref{fig:ISCT} depicts a general schematic of how an ISCT might run for nAMD drug testing.

\begin{figure}[t!]
  \centering
  \includegraphics[width=1\textwidth, height=9.3cm]{Fig_sec_8.png}
  \hfill
  \caption{Schematic of a generic ISCT with an example of nAMD drug testing. 1) \textit{In silico} model development of disease with validation loop linked to patient data. 2) \textit{In silic} model development of intervention with validation loop linked to \textit{in vitro} and \textit{in vivo}. 3) Virtual population generation using the disease model parameters. 4) ISCT run on intervention and control arms with appropriate outcomes and analysis.}
  \label{fig:ISCT}
\end{figure}

\subsection*{Mechanistic models of disease and intervention}

The central aspect of any ISCT is the mechanistic model that predicts the outcome of an intervention (or lack thereof). This is equivalent to the administration of the intervention in a real-world clinical trial. The mechanistic model is one that is based on the underlying physics and chemistry of the system being investigated. This model will usually simulate a disease, for example cancer tumour growth~\cite{Jenner2021}. The intervention is then also simulated, in this case oncolytic viruses that can target the tumour cells. Both the disease model and intervention model must be validated against \textit{in vitro} or \textit{in vivo} data in order to be used in an ISCT (Figure~\ref{fig:ISCT}).

Examples of disease/intervention models that have been developed include coronary stent models for occlusive heart diseases~\cite{Antonini2021, Berti2021}, insulin control algorithms in Type 1 diabetes~\cite{Kovatchev2009}, non-pharmaceutical interventions on respiratory tract interventions~\cite{Arsene2022}, bone morphogenetic treatment in paediatric orphan bone disease~\cite{Carlier2018}, anaemia treatment in haemodialysis patients~\cite{Fuertinger2018}, warfarin in atrial fibrillation patients~\cite{Ravvaz2017}, cancer vaccines on lymph node cancers~\cite{Gaffney2022}, targeted delivery of drugs in patients with covid-induced pneumonia~\cite{Wang2022}, mematine treatment of Alzheimer’s disease~\cite{Swietlik2022}, flow diverters in intracranial aneurysms~\cite{SarramiForoushani2021}, predicting pro-arrhythmic cardiotoxicity in cardiomyocytes~\cite{Passini2017}, and thrombectomy/thrombolysis for acute ischaemic stroke~\cite{Konduri2020}, amongst many more.

Within this review, we have also demonstrated the vast literature on \textit{in silico} models of retinal disease, ranging from non-neovascular AMD and retinitis pigmentosa (Section 6), to neovascular AMD and DR (Section 5), as well as models of intervention through intravitreal injections, implants, and sub-retinal injections (Section 7). This has laid the groundwork to adapt these models into the ISCT paradigm. Despite the rapidly growing literature on ISCTs in various disease conditions, little or no papers have been published on applying this paradigm to diseases of the retina.

Once a mechanistic model of disease and intervention has been established and extensively validated, the next step is to generate virtual populations with this disease that can be used in the ISCT.

\subsection*{Virtual Populations}

To run clinical trials, populations of the target population are required. A crucial first step is therefore to generate virtual populations (VPs) of the disease of interest that will eventually be used in the \textit{in silico} clinical trials. For example, if an intervention for nAMD patients will be tested \textit{in silico}, a population of nAMD patients is required.

Virtual population generation is a nascent and rapidly developing field. In essence, a virtual patient is one with a specific set of parameters for a given disease model. Virtual populations are therefore sets of patients with parameters that reproduce the statistics of interest of the clinical population of interest~\cite{Allen2016}.

One simple method of generating VPs is to assume a Gaussian distribution for each model parameter that will be patient specific~\cite{Gaffney2022, Jenner2021}. The mean and standard deviation of the parameters can be adjusted to match empirically observed data either manually or through optimization~\cite{Alfonso2020}. Often, these patient parameters are not independent of each other – for example, sex and height are correlated. Therefore, more complex sampling strategies that maintains the relationships between patient parameters have been developed.

Bayesian statistics have been used extensively to generate patient parameter populations for the purpose of ISCTs. Haddad et al. generated VPs using a Bayesian framework for augmenting clinical trials, demonstrating decreased sample size and trial length would be required for the real-world trial when using these VPs~\cite{Haddad2017a}. For warfarin patients, a Bayesian model was used to generate VPs from a pre-defined dataset~\cite{Fusaro2013}. Other methods that have been used to generate VPs include vine copulas to generate virtual stroke populations~\cite{Miller2021}. Pezoulas et al. also used tree-based methods (supervised and unsupervised) to generate VPs for cardiomyopathy drug development~\cite{Pezoulas2020} but found that Gaussian Mixture Models with Variational Bayesian inference outperformed supervised tree ensembles when comparing the VPs to the data~\cite{Pezoulas2021}.

As a virtual patient is effectively a set of parameters for the mechanistic model that represent a given individual, this can also be extended to 3-dimensional models of arteries or other physiological organs – where now the parameters might represent curvature of vessels, degree of stenosis, material properties of the artery etc. This introduces an added difficulty of modifying the mesh for each virtual patient. Because of this, most ISCTs involving patient meshes generate the mesh from patient-specific image data and simulate variability through variation in: orthopaedic implant positioning~\cite{AlDirini2019}; plaque growth over time on the arteries~\cite{Pleouras2021}; and blood pressure and thrombus formation parameters applied on the mesh~\cite{SarramiForoushani2021}.

VPs of retinal disease have yet to be extensively used in the literature. With appropriate data, however, VPs can be generated using the techniques documented.

\subsection*{Running the ISCT}

Once the mechanistic model and VPs have been developed, the ISCT protocol should be pre-defined – with a statistical plan, pre-defined primary and secondary outcomes, and trial inclusion and exclusion criteria. Current medical studies publish their protocols, with randomised clinical trials following the CONSORT guidelines~\cite{Schulz2010}. Once the study protocol is in place, the ISCT can be run and analysed with the pre-defined methodology.

Unlike real-world trials, the benefit of running an ISCT is the ability to run any permutation of trial you like, without consideration for cost or ethics. This allows for deeper investigation of the intervention in various sub-populations helping to optimise the application of the intervention.

The validity of the mechanistic model and ISCT is an important factor in translating this paradigm to bedside use. Guidelines have been published by the American Society of Mechanical Engineers, ''Verification \& Validation 40 Assessing Credibility of Computational Modeling through Verification and Validation: Application to Medical Devices''. These guidelines use a risk-informed credibility assessment, where the risk of the model defines how close the validation needs to be to real-world observations~\cite{ASME2018}.

Validation, verification, and uncertainty quantification (VVUQ) are necessary to give confidence in the models and ISCTs. As discussed, validation of the model depends on the context-of-use of the model and what risk is involved with its use, with higher risk models requiring higher validity~\cite{Pappalardo2019}. Validity of the VPs should also be considered (external validity) – the VPs should be representative of the wider population of interest for that disease i.e. not constrained by overly stringent exclusion criteria. Similarly, the ISCT should have ecological validity, where the results translate to real-life settings in which the trial results will be applied. Interesting research has looked at simulating hospital environments stochastically to determine ecological validity of the ISCT~\cite{Fuertinger2018}.

Verification is usually well established for given models and their numerical implementation~\cite{Curreli2021, Pappalardo2019}. Uncertainty quantification is also an essential step. This includes both aleatoric uncertainty (uncertainty in data used to inform model) and epistemic uncertainty (uncertainty in the knowledge of the physiological system used to build the model), as well as numerical uncertainty when solving approximated mathematical equations computationally. All of these uncertainties must be quantified to ensure a good comparison with real-world data can be made.


\section*{Discussion}

% \subsection*{Gaps in ISCTs for the retina}
Throughout this paper, we have reviewed the state-of-the-art models of the retina in both healthy and diseased conditions.

\textit{In silico} models have shed light on the interplay between biological mechanisms, such as autoregulation, necessary to maintain a healthy retina.
In addition, they have provided us with quantitative understanding of the range of insults that the retina can withstand.

In disease, the models brought insights on the underlying mechanics of common retinal disease, some of which may have been overlooked by traditional \textit{in vivo} and \textit{in vitro} investigation because of the difficulty of direct measurements.

Additionally, we have introduced the concept of \textit{in silico} clinical trials and the main components they require. 
Most of the building blocks for running an ISCT in the retina are already in place. When it comes to nAMD, there are already mechanistic models in place for the disease and intervention~\cite{Hoyle2017, Vega2021}. However, two main gaps still exist that are required to run successful ISCTs in retinal disease: 

\begin{enumerate}
\item{Patient data linkage for comprehensive VP generation. For representative and valid VPs, imaging data needs to be linked to primary and secondary health records to give a comprehensive disease population~\cite{ElBouri2021}.}

\item{Most clinical trials for retinal disease use visual acuity as a primary outcome, yet there is no \textit{in silico} model that reliably links visual acuity to the disease. This is the main stumbling block that needs to be addressed if ISCTs are going to be used extensively in retinal disease.}
\end{enumerate}

Whilst the future is incredibly promising for ISCTs, it is unlikely they will ever replace clinical trials. However, they can help refine and speed-up trials by eliminating pre-clinical testing and help refine the intervention to specific sub-populations or improve the intervention through repeated \textit{in silico} trial testing.



\bibliographystyle{abbrvnat}
\bibliography{bibliography}

\end{document}