\documentclass[12pt,a4paper]{article}
\usepackage[margin=1in]{geometry}

\usepackage[utf8]{inputenc}
\usepackage[T1]{fontenc}
\usepackage{amsmath}
\usepackage{amsfonts}
\usepackage{amssymb}
\usepackage{graphicx}
\usepackage{siunitx}

\graphicspath{{img/}}
\usepackage[numbers, square]{natbib}


\begin{document}

In proliferative DR and nAMD, loss of sight is linked to accumulation of fluids in the retina~\cite{Waldstein_2016, Roberts_2020}.
Those fluids are secondary to growth of leaky neovasculature in the neuronal retina and in particular in the macula.
The angiogenesis of macular neovasculature (MNV) is lead by gradients of vascular endothelial growth factors (VEGF) which are upregulated by the hypoxic conditions in the diseased retina.
The treatment of MNV is predominantly done via frequent injections of anti-VEGF agents that bind to the free VEGF present in the retina.
The drugs are normally injected inside the vitreous humour, in which they diffuse to reach the retina.
However, VEGF induced angiogenesis is a natural process meant to maintain the vasculature and remodel it to adapt it to its environment, both in the eye and the rest of the body.
Therefore, while effective for keeping visual acuity, the injections of pose a number of problems.
Firstly, the injections can cause further inflammations in the retina, triggering additional VEGF upregulation \textbf{[REF here]}.
Due to the distance travelled and the separation of vitreous humour and retina, only a fraction of the injected drug reaches the targeted site, usually in the outer retina~\cite{park_intraocular_2015}.
As a result, unbound anti-VEGF is found in significant levels in the systemic circulation, raising concerns about the safety of the intravitreal inejctions (IVI).
Indeed, while it is still matter of debate, it has been suggested that IVI could be linked with serious advert systemic effects including hemorrhages and strokes\textbf{[REF here]}. \\
Therefore, in the absence of other injection techniques, optimising the ratio of IVI injections to bioavailability in the retina of anti-VEGF is essential.
While \textit{in vitro} and \textit{in vivo} investigations of the pharmoacokinetics and pharmacodynamics of intravitreally injected anti-VEGF have been done, data on human is hardly available.
Mathematic




\begin{itemize}
\item 
\end{itemize}


% \begin{spacing}{0.0}
\bibliographystyle{abbrvnat} % {ksfh_nat}
{\normalsize \bibliography{Anti-VEGF}}
% \end{spacing}

\end{document}
