\documentclass[12pt,a4paper]{article}

\usepackage[utf8]{inputenc}
\usepackage[T1]{fontenc}
\usepackage{amsmath}
\usepackage{amsfonts}
\usepackage{amssymb}
\usepackage{graphicx}
\usepackage{authblk}

\usepackage[super,numbers]{natbib}

\usepackage{tcolorbox}
\usepackage{siunitx}

\begin{document}

The highly detailed images our eyes are capable of capturing necessit the close interplay of many different cells and structures.
Degradation of visual functions can severely affect the quality of life of patients.
The retina is a complex and fragile tissue playing a major role in visual functions.
As such, it is perhaps not surprising that many severe visual impairments find their root in the retina.

\begin{itemize}
\item Imaging techniques, recent development of noninvasive devices
\item (still) Clinical care rely on observations of symptoms
\item Standardized therapies showed limitations
\item Little to no information on the state of the retina and causes of disease in invidual patients (e.g., hypoxia, cellular functions, weakness of the vasculature...)
\item Increase in use of mathematical/\textit{in silico} models for basic research in the past two decades by providing insights that are hard or impossible to achieve by experimental or observational means
\item Increasing interest in using \textit{in silico} evidence to inform treatment and clinical trials
\item The recent increase in the number of literature reviews on mathematical models of the eye and retina (cite them) and the development of platforms for simulating the eye accessible to non-modellers (cite Ocular mathematical virtual simulator) are evidence of that interest.
\item \textit{In silico} clinical trials can enhance both clinical care and traditional clinical trials, preserving sight in the fast growing number of patients suffering from retinal disorders.
\item \textit{In silico} trials require close collaboration between experimentalist, modellers and clinicians (see Figure Paul).
\item \textit{In silico} trials have been successfully used in other medical discipline, in particular cancerology and cardiology.
\item Ophthalmology can also benefit from computer aided clinical development.
\item Furthermore, the eye provides researchers with a wealth of images and direct measurements that is perhaps unparalleled in other medical specialties.
\item Breakdown of the paper:
  \begin{itemize}
  \item An overview of current models of the retina in health and in disease, to provide the reader with an understanding of the benefits.
  \item Additionally, models of common therapies will also be summarized.
  \item Following the overview of state-of-the-art models, we provide a breakdown of what constitute \textit{in silico} trials and present examples of successful applications of such trials.
  \item Finally, we conclude by highlighting the gaps that need to be closed before running \textit{in silico} trials for the retina.
  \end{itemize}
\end{itemize}

\bibliographystyle{abbrvnat} % {ksfh_nat}
\bibliography{../bibliography}

\end{document}