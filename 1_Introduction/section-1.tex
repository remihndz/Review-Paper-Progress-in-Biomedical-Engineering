\documentclass[12pt,a4paper]{article}

\usepackage[utf8]{inputenc}
\usepackage[T1]{fontenc}
\usepackage{amsmath}
\usepackage{amsfonts}
\usepackage{amssymb}
\usepackage{graphicx}
\usepackage{authblk}

\usepackage[super,numbers]{natbib}

\usepackage{tcolorbox}
\usepackage{siunitx}

\begin{document}

The highly detailed images our eyes are capable of capturing necessit the close interplay of many different cells and structures.
Degradation of visual functions can severely affect the quality of life of patients.
The retina is a complex and fragile tissue playing a major role in visual functions.
As such, it is perhaps not surprising that many severe visual impairments find their root in the retina.

\begin{itemize}
\item Development of imaging techniques
\item Clinical care rely on observations of symptoms
\item Little to no information on the state of the retina and causes (e.g., hypoxia, cellular functions, weakness of the vasculature...)
\item Increase in use of mathematical/\textit{in silico} models for basic research in the past two decades by providing insights into systems that are hard or impossible to investigate by experimental or observational means alone
\item Increasingly more interest in using \textit{in silico} evidence to inform treatment and clinical trials
\item As a proof of that interest is the recent growing body of literature reviews on mathematical models of the eye and retina
\item \textit{In silico} clinical trials can enhance clinical care of retinal diseases
\item \textit{In silico} trials require close collaboration between experimentalist, modellers and clinicians 
\item 
\end{itemize}

\bibliographystyle{abbrvnat} % {ksfh_nat}
\bibliography{../bibliography}

\end{document}