\documentclass[12pt,a4paper]{article}
\usepackage[utf8]{inputenc}
\usepackage[T1]{fontenc}
\usepackage{amsmath}
\usepackage{amsfonts}
\usepackage{amssymb}
\usepackage{graphicx}

\begin{document}

\subsection{Retinal haemodynamics}
\begin{itemize}
\item Intro to what is modeling haemodynamics and the mechanisms at play
\item What diseases and treatment can benefit from modeling this
\item Basis of modelling haemodynamics (i.e. Poiseuille flows, networks, Navier-Stokes, external pressures, Murray's law....)
\end{itemize}
Retinal haemodynamic models are concerned with describing blood flow within the retinal circulation.
Adequate blood flow is essential to ensure delivery of nutrients and oxygen to the retinal cells as well as removing the waste products of those same cells.
Retinal blood flow is controlled by various mechanisms of auto-regulation in order to adapt to changes in the systemic blood flow or in the retinal or choroidal environment.
In response to such changes, retinal vessels can adjust their caliber by making use of a number of communication pathways.
These pathways can be triggered by either a change in metabolic demand of the tissue or variations in pressure.
In particular, vessels are sensitive to variations in the difference between intraocular pressure (IOP) and blood pressure (BP).
Increases in IOP might cause collapse of vessel walls and therefore local shutdowns of blood flow.
Elevated IOP is related with diseases such as glaucoma and retinal vein occlusion. 
Additionally, haemodynamic parameters are direct determinants of oxygen delivery to the tissue.
Therefore, retinal haemodynamic is also relevant to diseases such as neovascular AMD and diabetic retinopathy and other hypoxia-induced disorders. \\
While direct \textit{in vivo} pressure measurements are difficult to achieve, mathematical models can help understand the relationships between retinal haemodynamics and systemic observations (e.g., mean arterial pressure and IOP measurements).
For the most part, those models rely on a framework consisting of:
\begin{itemize}
\item A network of fully connected vascular segments
\item A law describing the viscosity of blood as a function of a vessels diameter
\item A law describing blood flow as a function of viscosity and vessel geometry
\end{itemize}

\subsubsection{Healthy retina}
\begin{itemize}
\item Takahashi's network and distribution of haemodynamics parameters
\item Alletti 2016, transient, Navier-Stokes, fluid-structure interactions
\item Zouache 2015 Choriocapillaris modelling: talk about the missing validation
\end{itemize}

To the best of our knowledge, the first haemodynamic model of the retinal microcirculation is a work done by Takahashi et al.~\ref{Takahashi2009}.
Their aim was to provide a simple distribution of haemodynamic parameters in the human retinal microvasculature.
In this work, the artiolar tree is approximated by a dichotomous branching tree, with each generation of vessels branching into two vessels, starting from the central retinal artery.
Radius of the daugther vessels is determined by Murray's theoretical law:
\begin{equation}
  \label{eq:MurrayLaw}
  r_p^\gamma = r_{d,1}^\gamma + r_{d,2}^\gamma
\end{equation}
with $r_p$ the parent vessel radius and $r_{d,i},~i=1,2$ the radius of the daugther vessel~\ref{Murray1926}.
The exponent $\gamma$ was determined theoretically by Murray to be equal to $3$, however Takahashi et al. used $\gamma=2.85$ as suggested by more recent work.
After branching dichotomously fourteen times, each of the terminal arterioles sprout four capillaries of fixed homogeneous calibre. 
The venous tree symetrically, starting from the central retinal vein to reach the capillary level where both trees are linked together.

Blood is assumed to be an incompressible Newtonian fluid.
Since the vessels are assumed cylindrical and long with a constant cross section for each generation, the Poiseuille law can be used to describe blood flow.
Poiseuille law describes blood flow as proportional to the pressure drop along the vessel by the relation:
\begin{equation}
  \label{eq:PoiseuilleLaw}
  Q = \frac{\pi r^4}{8\mu L}\Delta p
\end{equation}
where $Q$ is the blood flow, $r$ is the vessels radius, $L$ the vessel length, $\mu$ the blood viscosity and $Delta p$ the pressure drop along the vessel's length.
This approximation fails to work in the smallest calibre vessels of diameter equal or smaller than of the diameter of the red blood cells flowing through them.
As a consequence, vascular resistance is strongly hightened.
To account for this effect, the viscosity of blood is replaced by a varying effective viscosity $\mu_e = \mu_e(r)$ which accounts for this effect and the so-called F\r{a}hr\ae us-Lindqvist effect.
Such viscosity law is only phenomenological and increases viscosity of blood sharply as the vessel radius reaches the size of red blood cells.\\
This work provides an initial estimate of the distribution of haemodynamic parameters in the arterio-venous network in the retina.
It was validated geometrically by comparing the aspect ratio at branches of the arteriolar tree to data from fluorescein photographs of the retina. 
However, it is likely that its efficiency is higher than actual networks since it is generated based on optimality principles and does not include the influence of variying external and internal pressures on the blood flow.



\subsubsection{Ageing and diseased retina}
\begin{itemize}
\item Rebhan 2019: Glaucoma and diabetic retinopathy, haemodynamics and tissue strees
\item Flower 2001: choroidal blood flow and treatment of subfoveal neovsacularization (Might not be the best place to put it?)
\end{itemize}

\subsection{Anti-VEGF therapy}
\subsubsection{Laser treatment}


\end{document}